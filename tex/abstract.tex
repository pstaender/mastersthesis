\section{Introduction}

\subsection{Motivation and Research Questions}

Open source software (OSS) is gaining more and more importance as innovation factor in software businesses. Also the number of firms who initiated open source projects has increased quite substantially over time.  Therefore OSS has reached increasing attention by scholars in fields of organization and open innovation over the last two decades.

There are various popular examples of open source projects which demonstrate the feasibility of running a business with or on open source projects. Google, Twitter, facebook, Microsoft and IBM are just some well known firms who initiated open source projects with remarkable success. Of course not only big companies are actively developing OSS. There are also a variety of small and medium-sized enterprises that have successfully initiated and run open source projects (like Neo4j or MongoDB),  where most of the open source projects are also part of their business.

Open source communities are technical communities and can help firms to develop and deploy new technical innovations (Rosenkopf, Metiu and George, 2001; Rosenkopf and Tushman, 1998; Mowery and Simcoe, 2005; Fleming and Waguespack, 2007). So firms have a reason to collaborate with open source communities. Nonetheless firms that are involved in open source projects have to face various difficulties. One important point is that firms have to care about their relationship to the open source community and have to find a fair way to join commercial (of the firm) and non-commercial interests (of the open source community) without reducing the efficiency and the quality of innovation. Because of that symbiotic relation between the firm and the open source community (Dahlander and Magnusson, 2005) organizational obstacles have to be solved by a firm if it wants to collaborate with autonomous open source communities (Piezunka and Dahlander, 2013) or sponsored open source communities (West and O’Mahony, 2008).

This leads to the question to what extent affects a firm's internal resources on external resources. If we define external contributors as individuals who contribute their own ideas (Piller and Walcher, 2006), solutions (Jeppesen and Lakhani, 2010), knowledge (Laursen and Salter, 2006) or innovations (von Hippel, 1988) we can regard external developers, who are external contributors, as a resource of knowledge. Thus, external developers will affect the capability of firms’ absorptive capacity of prior related knowledge (Cohen and Levinthal, 1990). Furthermore we can assume that external developers may have similar project-related knowledge as firm employed developers over time (Piezunka and Dahlander, 2013) and do represent their own interests where firms employed developers (sponsored contributors) represent the firm’ s interests (Dahlander and Wallin, 2006).

\subsection{How can the success of a firm-initiated open source project be measured?}

Firms are driven by economic and technological factors (Feller and Fitzgerald, 2002) whereas developers are more driven by social motivation factors (Bonaccorsi and Rossi, 2003). So in between we have the technological motivation factor which motivates firms and developers. So we can roughly divide success into economic and social success. Economic success of a project (for a firm) is more easy to define, social success requires a more precise definition. Social success in our case means popularity in open source communities. Be it that many developer pay attention to the project or actively participate on the project. As shown in previous research the number of participants depends on the complexity of the open source project (Dahlander and Piezunka 2013). In our case we will concentrate on social success of open source projects. They can be analyzed in a more reliable way since most firms simply do not release any business related numbers on their projects for external research.

To compare the impact of internal and external developers on a firm initiated open source project we have to analyze the kind and rate of contribution. If we narrow down the process of software development to (1) participation by submitting source code and (2) taking part on discussions and support channels (also called interaction tools) we can regard (1) as technological and (2) as social contribution. Social contribution can be proceed in submitting and discussing ideas on newsgroups or solving problems on support forums. Thus we should be able to derive a measure of success by these two participation possibilities.

As mentioned above this study will note analyze the economic success of firm initiated open source project since key data of revenue and costs are not available for the most analyzed open source projects. Nevertheless I want to point out that firms have various capabilities to monetize their initiated open source project. Many firms do provide extended support and offer professional services about their open source projects. To name some examples of firms running businesses on their initiated open source projects are MySQL 1 (acquired in 2010 by Oracle Corporation 2), Neo4j 3, MongoDB 4  and Zend 5.

The following research questions are defined for problem description:

\begin{itemize}
	\item RQ1: To what extent does firm employees’ participation affect the success of firm-initiated open source projects
	\item RQ1.1: How does firm employees’ participation affect the success by contributing source code?
	\item RQ1.2: How does firm employees’ participation on communication channels (i.e. interaction tools, see below) affect the success?
	\item RQ1.3: How many (internal and external) users participate and how long and how often over time?
	\item RQ1.4: How many (internal and external) suggestions are submitted and finally merged into the project?
\end{itemize}

In our case we will focus on web technology software. Therefore we will exclusively analyze firms with major open source projects which primarily provide web technology software.

The measurement of contributions’ quality will be challenging. There are several reasons why a contribution of source code is not being merged into the final project. The committed code can be (1) buggy, (2) not useful, (3) or requires further discussion. Similar but even more difficult is to measure quality of contribution on interaction tools. This originates from the fact that finding and implementing new ideas is a complex and subjective process which becomes the more difficult the higher the numbers of participants are.

\subsection{Theory and Prior Research}

There are several reasons why firms initiate open source projects. As West (2003) noticed firms do publish source code in order to get their product widely adopted to increase the likelihood to attract developers (first-mover advantages) and achieving faster technological development (Lerner and Tirole, 2002). Beside that all open source projects can be termed as collective action model which applies to the provision of public goods, where a public good is non-excludable and non-rival (Olson 1967, p. 14), so that open source projects can be regarded as basis for a novel, private-collective model for the motivation of innovation (von Hippel and von Krogh, 2003). This agrees to assumption that any interested person can voice their opinion when organizations search for innovation by soliciting suggestions from externals (Alexy et al., 2012b). Because most of the firm initiated open source projects are part of the firm's business, they have to utilize resources as knowledge and ideas to maximize their value creation and their profit in the long run (Dahlander and Magnusson, 2005).

Because firm initiated open source projects depend on the symbiotic between internal end external developers, externals’ suggestions can lead organizations to novel, useful, and actionable ideas (Hill \& Birkinshaw, 2010), develop new concepts (Ahuja \& Katila, 2004; Jeppesen \& Lakhani, 2010; Shane, 2000; Un, 2011) or find new markets for their technologies (Gruber, MacMillan, \& Thompson, 2008; Shane, 2000). So the “wisdom of the crowds” (Surowiecki, 2005) depends external suggestions which facilitates research on external sources of innovation (Piezunka and Dahlander, 2005) as on open innovation (Chesbrough, 2003), crowdsourcing (Afuah \& Tucci, 2012) and user-based innovation (von Hippel, 1986).

Dahlander and Piezunka showed that most initiatives to engage external contributors do fail (Dahlander and Piezunka, Open to suggestions: How organizations elicit suggestions through proactive and reactive attention, 2013). They developed arguments about what increases the likelihood of getting suggestions from externals by proactive and reactive attention to suggestions. Moreover they say that firms and communities have divergent rationales for existing, which causes problems on interaction.
They found out that early organizations’ engagements weigh more in the early stage and that firms should pay attention to new contributors. Beside that they suggest to keep barriers to entry low because this results in lower threshold for participation.

In 2013 Dahlander and Piezunka (A study of Organizations’ attention to suggestions by externals over time, 2013) developed a theory about the structures that are build around a suggestion after it is posted. They say the more suggestions are related to each other, the greater the externals’ demand  for a suggestion and the more a suggestion is debated by externals, the higher the likelihood an organization will attend to it. On the other side the greater the diversity of support a suggestion attracts and the more a suggestion competes with each other, the lower the likelihood that an organization will attend to it.


Drawing 1: Decision Process (A study of organizations' attention to suggestions by externals over time, Piezunka, Henning and Dahlander, Linus, 2013)
One of the initial ideas of the OSS community is to work for a non-commercial reasons. This means that firms have to deal with the term openness (Dahlander and Gann, 2010) and have to accept that external developers are free to join and work on informal relationships (Dahlander and Magnusson, 2005) whereas firm-based software creation is normally restricted to relations within the firm. To use the taxonomy by Feller and Fitzgerald (2002) we can distinguish between economic, social and technological motivation factors. As Bonaccorsi and Rossi (2003) found out that firms are driven by economic and technological factors rather than by social motivation factors. External developers contribution depends on both extrinsic and intrinsic motivation (Hertel et al., 2003; Hars and Ou, 2002; Lakhani et al., 2002), which is a social motivation, or even by intellectual challenges (Raymond, 1999; Hertel et al., 2003; Lakhani et al., 2002).  This agrees with the assumption that users often find solutions to their own problems and are willing to share when the marginal cost of sharing is low (von Hippel, E., 1976). Beside that achievements of users in open source projects are not protected by intellectual property rights (Waguespack and Fleming, 2004) which is a great contrast to intellectual property handling of traditional internal firm software development.

Joel West and Siobhan O’Mahony analyzed in their study from 2008 (The Role of Participation Architecture in Growing Sponsored Open Source Communities) that corporate sponsorship affects how open source communities are designed and evolve. They identified three design dimensions for corporate sponsors when designing open source communities: 1) intellectual property rights, 2) development approach, and 3) model of community governance. Overall they came to the conclusion that openness of sponsored open source projects are more likely to offer transparency than accessibility, which has implications for their communities’ growth.

In 2005 Dahlander and Magnusson (Relationships between open source software companies and communities: Observations from Nordic firms) analyzed also the relationship between open source software firms and communities. They found out that firms who depend on a symbiotic approach have subtle means of control to influence the community and are confronted with challenging managerial issues. For instance, if a firm is well-known and respected in the community, they have more influence on the development activities performed in the community compared to less well- connected ones.
Generally they distinguish five mechanism of subtle means of control:

\begin{itemize}
	\item devote firm employees to work with and in the community
	\item Reputation of firm employees in the community
	\item Fringe benefits
	\item Interaction tools which offers communication channels for the community
	\item Provide interesting tasks ('selling' development tasks)
\end{itemize}

Greater possibility of influencing the community might result in several benefits, but may also emerge in the following managerial issues:

\begin{itemize}
	\item respecting the norms and values of OSS communities
	\item using licenses in a suitable way
	\item attracting developers and users
	\item dealing with the resource consumption involved in community development
	\item aligning different interests about the nature of work
	\item resolving ambiguity about control and ownership
\end{itemize}

The mechanisms and managerial issues will be also essential reference points for this research.

\subsection{Empirical Design}

As mentioned before the study will concentrate on participation of firm employed employees and external developers on web technology software providing open source projects.

We will use GitHub 6 as main data source for the research. GitHub is very popular in the web technology open source community 7 and provides a suitable API 8 for data analysis 9. Therefore we will pool users and developers to the term “user” because every user on GitHub is also a developer by default and can't be distinguished.

There are several small, medium and large enterprises that have initiated open source projects which are part of their business and accepted by the community. Here is a list of firms which fulfill the criteria and would be interesting to investigate (from small to large):

\begin{itemize}
	\item SilverStripe 10
	\item Couchbase 11
	\item OwnCloud 12
	\item Neo4j 13
	\item ElsasticSearch 14 and Graylog2 15
	\item MongoDB 16
	\item Groupon 17
	\item Facebook 18
	\item Google 19
	\item Microsoft 20
\end{itemize}

The selection of companies and projects is focused on web software technology in this study, as mentioned before. This includes DBMS 21 that are essential for web technology and web business. Neither the form of company (LTD, INC …) nor the form and type of licenses is relevant for our study.

According to the research questions we will analyze contributions of source code (commits) and usage of communication (bug reporting and other general issues), which both are provided by the hosting platform. Submitting commits and participating on issues can be regarded as submitting a suggestion (Dahlander et al. 2013). In order to get comparable data we will use exclusively GitHub as data source and will not cover other communication channels as newsgroups, forums, CRM tools or other project management software.

To measure the popularity of open source projects for non-active participating users, we can easily count stargazers (people who favor a project) and watchers (people who monitor the activity of a project).

Any firm listed above maintains a company account on GitHub where we can analyze the open source projects. Moreover we will be able to match firm employed developers and external developers and may gain a quite reliable ratio number.

Finally we will have to compare all projects to each other in order to receive a meaningful view on the success and the role of firm employees on the firm initiated open source project.

\subsection{Expected Results and Implications}

As mentioned before there might be difficulties to measure the actual quality for a contribution, especially on interaction tools and communication channels respectively. Since source code submissions are easier to proof as a useful contribution the evaluation on merged and rejected source code  could lead to a better assessment of the project’s activity, efficiency and success.

We can assume that the higher the number of participations of internal and external developers are, the higher the likelihood that the project is successful and popular.

The rate of external developers and firm-employed developers on a project in context to its success will be quite interesting to evaluate. As found out in earlier studies (see Theory and prior research for more details) the reasons why and whether externals and internals suggestions are finally merged into the project are versatile. This originates from the fact that the organization and participation by a firm on an open source project is a complex interplay between two groups (a firm with internal developers and the open source community with external developers), incentives and goals.

Software projects are based on concepts which are fundamental for their purpose. In web technology business concepts may change over time because the web (and it's technology) is in continuous development and change. That means for a web software project requirements do change steadily as well. So this could also be a crucial factor for influencing internal and external commitment on a project.

Not all successful and popular open source have necessarily the most active communities. One reason is that accepted open source projects that run for considerable time have found their design, purpose and stability. So changes and number of new features might be marginal without declining the quality of the open source project. Moreover does social success (as analyzed in this study) not imply by default economic success. There are several open source projects with remarkable commercial success that have smaller and less active communities and vice versa. One reason for that are free riders (von Hippel, Eric and von Krogh, Georg, 2003), people who download and use the software without any contribution.

To analyze the social success of a firm initiated open source projects on the global open source community (i.e. all social platforms that are somehow connected to open source contribution) would be a huge effort and can not be covered in this study. I guess that the diversity of communication channels and the abundance of information about firms open source commitment makes it difficult for smaller enterprises to advertise their projects in contrast to well known enterprises. Thus, the size and budget is also a factor of the likelihood that a firm initiated open source project gains success. So the higher the size and budget of a company, the more employees’ are able to participate and maintain the firm’s open source project.

After all we can assume that the health and success of a firm initiated open source project depends more on the participation of firm employed developers than on external input. This results from the fact that any firm which initiates an open source project should have a concrete idea of its purpose and impact on their business (von Hippel, Eric and von Krogh, Georg, 2003).  Therefore firm employed developers would maintain and participate on the open source project to support the initial idea of the firm. Especially if the community is hard to motivate in the long run, participation of firm employed developers can stabilize and push the development of project steadily.
