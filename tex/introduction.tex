\section{Introduction}

Open source software (OSS) is gaining more and more importance as innovation factor in software businesses. Also the number of firms who have initiated open source (OS) projects has increased quite substantially over time. Therefore, OSS has reached increasing attention by scholars in fields of organization and open innovation over the last past three decades.

There are various popular examples of OS projects  which demonstrate the feasibility of running businesses with or on initiated OS projects (Google, Twitter, facebook and Microsoft are just a few popular examples). But not only big companies are actively involved in OSS development. A variety of small and medium-sized enterprises have also initiated and maintained OS projects successfully. Some of those OS projects are also part of their core business.

OS communities are technical communities and can help firms to develop and deploy new technical innovations (\cite{rosenkopf2001bottom}, \cite{rosenkopf1998coevolution}, \cite{simcoe2006public}; \cite{fleming2007brokerage}). Thus, firms have credible reasons to collaborate with OS communities. Nonetheless firms that are involved in OS projects have to face various difficulties. One important point is that firms have to care about their relationship to the OS community and have to find a fair way to join commercial (of the firm) and non-commercial interests (of the OS community) without reducing the efficiency and the quality of innovation. Because of this symbiotic relationship between the firm and the OS community (\cite{dahlander2005relationships}) organizational obstacles have to be solved by a firm if it wants to collaborate with autonomous OS communities (\cite{piezunka2013study}) or sponsored OS communities (West and O’Mahony, 2008).

This leads to the question to what extent firm's internal resources affect external resources. If we define external contributors as individuals who contribute their own ideas (\cite{piller2006toolkits}), solutions (\cite{jeppesen2010marginality}), knowledge (\cite{laursen2006open}) or innovations (\cite{urban1988lead}) we can regard external developers, who are external contributors, as a resource of knowledge. Thus, external developers will affect the capability of firm's absorptive capacity of prior related knowledge (\cite{cohen1990absorptive}). Furthermore, we can assume that over time external developers may have similar project-related knowledge as firm employed developers (\cite{piezunka2013study}) and do represent their own interests whereas firms employed developers (sponsored contributors) represent the firm’ s interests (\cite{dahlander2006man}).

As \cite{dahlander2005relationships} found out, "OSS firms can use symbiotic, commensalistic, or parasitic approaches for interrelating to their communities" and by using the "symbiotic approach, firms have more possibilities to influence the (OS) community" \cite[p.491-492]{dahlander2005relationships}. A new study from 2014 shows that "OSS projects with large peripheral developer participation \footnote{peripheral developers = external developers} are those that are likely to be innovative and offer high-quality" \cite[p.40:22]{krishnamurthy2016peripheral} and that "presence of feature requests is positively associated with peripheral developer participation" \cite[p.40:23]{krishnamurthy2016peripheral}.

This study will focus on the impact of firm developers' commitment on external developers' commitment and the long-term success of the projects. Therefore 58 firms with more than 3000 published projects (written in the 10 most popular programming languages) hosted on GitHub \footnote{the worldwide largest and most popular open source hosting service (see \cite{GitHubDominatesTheForges:online}; \cite{HowGitHubConqueredGoogleMicrosoftAndEveryoneElse:online}; \cite{AboutGitHubPress:online})} will be studied. We will measure code contribution and communication actions (participation on issues) referred to (1) the influence of firm employed developers on external developers (and v.v.) and (2) the final success / acceptance of a project on the platform. Thus, the main questions are \footnote{the hypotheses are formulated in chapter \ref{sec:hypotheses} on page \pageref{sec:hypotheses}}:

\begin{itemize}
  \item To what extent does firm employed developers' commitment encourage external developers' commitment?
  \item Do external developers more often contribute source code if firm employed developers contribute proportionate more than usual?
  \item Do external developers participate more actively in issues if firm's developers do as well?
  \item Does participation in the beginning affect the later success of firms' initiated projects?
\end{itemize}

The empirical findings mainly confirm all hypotheses to the effect that participation of firm employed developers has a significant positive impact on external commitment and long-term success of projects. Further details, explanations and interpretations of the empirical findings can be found in chapter \ref{sec:results} on page \pageref{sec:results}.
