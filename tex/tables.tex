\section{Tables}

\subsection{Introduction and Theory}

\begin{table}[ht]
\centering
\begin{tabular}{rllll}
  \hline
 & Linux & FreeBSD & Unknown & MS.Windows \\
  \hline
W3Techs (02-2015) & 35.9\% & 0.95\% & 30.9\% & 32.3\% \\
  W3cook (05-2015) & 96.6\% & 1.7\% & 0\% & 1.7\% \\
   \hline
\end{tabular}
\caption{Share of Operating Systems on public Internet Servers; \textit{Sources: \url{http://w3techs.com/technologies/overview/operating_system/all}, \url{http://w3techs.com/technologies/details/os-unix/all/all}, \url{http://www.w3cook.com/os/summary/}}}
\label{tbl:publicserver2015}
\end{table}

\begin{table}[!h]
\centering
% Table created by stargazer v.5.2 by Marek Hlavac, Harvard University. E-mail: hlavac at fas.harvard.edu
% Date and time: Sat, Mar 26, 2016 - 18:58:30
% \begin{table}[!htbp] \centering
%   \caption{}
%   \label{}
\begin{tabular}{@{\extracolsep{5pt}}lccccc}
\\[-1.8ex]\hline
\hline \\[-1.8ex]
Statistic & \multicolumn{1}{c}{N} & \multicolumn{1}{c}{Mean} & \multicolumn{1}{c}{St. Dev.} & \multicolumn{1}{c}{Min} & \multicolumn{1}{c}{Max} \\
\hline \\[-1.8ex]
Top Project (yes/no) & 177 & 0.181 & 0.386 & 0 & 1 \\
Ratio & 177 & 0.684 & 0.334 & 0.000 & 1.000 \\
Age (days) & 177 & 294.260 & 226.632 & 20 & 1,466 \\
Number of Commits & 177 & 1,241.955 & 10,558.430 & 3 & 127,063 \\
Firm Employees' Commits & 177 & 138.073 & 307.597 & 0 & 2,573 \\
External Developers' Commits & 177 & 1,103.881 & 10,548.850 & 0 & 126,912 \\
Stars & 177 & 186.051 & 580.191 & 0 & 4,909 \\
Contributors & 177 & 9.949 & 27.287 & 2 & 262 \\
Subscribers & 177 & 50.655 & 71.049 & 2 & 585 \\
Forks & 177 & 45.153 & 155.701 & 0 & 1,812 \\
Number of Issues & 177 & 15.429 & 43.100 & 1 & 420 \\
\hline \\[-1.8ex]
\end{tabular}
% \end{table} 

\caption{Statistic of Projects by Microsoft}
\label{tbl:summary_microsoft}
\end{table}

\begin{table}[!h]
\centering
% Table created by stargazer v.5.2 by Marek Hlavac, Harvard University. E-mail: hlavac at fas.harvard.edu
% Date and time: Sat, Mar 26, 2016 - 18:58:04
% \begin{table}[!htbp] \centering
%   \caption{}
%   \label{}
\begin{tabular}{@{\extracolsep{5pt}}lccccc}
\\[-1.8ex]\hline
\hline \\[-1.8ex]
Statistic & \multicolumn{1}{c}{N} & \multicolumn{1}{c}{Mean} & \multicolumn{1}{c}{St. Dev.} & \multicolumn{1}{c}{Min} & \multicolumn{1}{c}{Max} \\
\hline \\[-1.8ex]
Top Project (yes/no) & 40 & 0.400 & 0.496 & 0 & 1 \\
Ratio & 40 & 0.528 & 0.333 & 0.000 & 1.000 \\
Age (days) & 40 & 1,039.925 & 615.875 & 62 & 2,449 \\
Number of Commits & 40 & 1,784.450 & 7,037.635 & 5 & 44,486 \\
Firm Employees' Commits & 40 & 290.175 & 516.544 & 0 & 1,905 \\
External Developers' Commits & 40 & 1,494.275 & 7,021.648 & 0 & 44,482 \\
Stars & 40 & 993.175 & 1,705.672 & 1 & 7,922 \\
Contributors & 40 & 56.625 & 114.295 & 2 & 451 \\
Subscribers & 40 & 62.750 & 62.040 & 6 & 233 \\
Forks & 40 & 244.350 & 518.172 & 1 & 2,611 \\
Number of Issues & 40 & 20.500 & 43.776 & 1 & 242 \\
\hline \\[-1.8ex]
\end{tabular}
% \end{table}

\caption{Statistic of Projects by GitHub}
\label{tbl:summary_github}
\end{table}

\begin{table}[!h]
\centering
% Table created by stargazer v.5.2 by Marek Hlavac, Harvard University. E-mail: hlavac at fas.harvard.edu
% Date and time: Sat, Mar 26, 2016 - 18:57:38
% \begin{table}[!htbp] \centering
%   \caption{}
%   \label{}
\begin{tabular}{@{\extracolsep{5pt}}lccccc}
\\[-1.8ex]\hline
\hline \\[-1.8ex]
Statistic & \multicolumn{1}{c}{N} & \multicolumn{1}{c}{Mean} & \multicolumn{1}{c}{St. Dev.} & \multicolumn{1}{c}{Min} & \multicolumn{1}{c}{Max} \\
\hline \\[-1.8ex]
Top Project (yes/no) & 103 & 0.534 & 0.501 & 0 & 1 \\
Ratio & 103 & 0.591 & 0.342 & 0.000 & 1.000 \\
Age (days) & 103 & 731.204 & 517.623 & 79 & 2,489 \\
Number of Commits & 103 & 1,103.039 & 4,112.523 & 5 & 36,596 \\ 
Firm Employees' Commits & 103 & 559.748 & 1,682.788 & 0 & 14,499 \\
External Developers' Commits & 103 & 543.291 & 3,553.457 & 0 & 35,872 \\
Stars & 103 & 2,360.272 & 4,762.077 & 12 & 35,214 \\
Contributors & 103 & 36.767 & 70.711 & 2 & 468 \\
Subscribers & 103 & 188.990 & 347.568 & 16 & 2,617 \\
Forks & 103 & 396.107 & 854.878 & 2 & 5,716 \\
Number of Issues & 103 & 56.029 & 161.550 & 1 & 1,043 \\
\hline \\[-1.8ex]
\end{tabular}
% \end{table}

\caption{Statistic of Projects by facebook}
\label{tbl:summary_facebook}
\end{table}

\clearpage
\subsection{Data and Basic Statistics Tables}

	% latex table generated in R 3.2.2 by xtable 1.8-0 package
% Tue Feb  9 23:24:08 2016
{\footnotesize
\begin{longtable}{rlrl}

  \hline
 & Name of Organization on GitHub & Top Repositories & is commercial Organization \\
  \hline
  \endhead
  1 & google &  85 & yes \\
    2 & facebook &  56 & yes \\
    3 & Microsoft &  33 & yes \\
    4 & square &  30 & yes \\
    5 & apache &  23 & partly \\
    6 & aspnet &  23 & partly \\
    7 & thoughtbot &  23 & yes \\
    8 & alibaba &  19 & yes \\
    9 & mozilla &  18 & partly \\
    10 & twitter &  18 & yes \\
    11 & Netflix &  17 & yes \\
    12 & github &  16 & yes \\
    13 & mongodb &  16 & yes \\
    14 & mono &  16 & no \\
    15 & thephpleague &  16 & no \\
    16 & docker &  15 & yes \\
    17 & elastic &  15 & yes \\
    18 & hashicorp &  15 & yes \\
    19 & golang &  14 & partly \\
    20 & Yalantis &  13 & yes \\
    21 & dotnet &  12 & partly \\
    22 & rails &  12 & partly \\
    23 & airbnb &  10 & yes \\
    24 & aws &  10 & yes \\
    25 & etsy &  10 & yes \\
    26 & googlesamples &  10 & yes \\
    27 & linkedin &  10 & yes \\
    28 & xamarin &  10 & yes \\
    29 & zeromq &  10 & no \\
    30 & coreos &   9 & partly \\
    31 & laravel &   9 & no \\
    32 & Shopify &   9 & yes \\
    33 & spring-projects &   9 & partly \\
    34 & adafruit &   8 & yes \\
    35 & bitly &   8 & yes \\
    36 & dmlc &   8 & no \\
    37 & doctrine &   8 & no \\
    38 & facebookarchive &   8 & partly \\
    39 & fastlane &   8 & no \\
    40 & jquery &   8 & no \\
    41 & KnpLabs &   8 & yes \\
    42 & shadowsocks &   8 & no \\
    43 & Yelp &   8 & yes \\
    44 & apple &   7 & yes \\
    45 & Automattic &   7 & yes \\
    46 & Azure &   7 & yes \\
    47 & cocos2d &   7 & no \\
    48 & couchbase &   7 & no \\
    49 & douban &   7 & yes \\
    50 & FriendsOfPHP &   7 & no \\
    51 & id-Software &   7 & yes \\
    52 & libgit2 &   7 & no \\
    53 & stripe &   7 & yes \\
    54 & cloudera &   6 & yes \\
    55 & dropbox &   6 & yes \\
    56 & enormego &   6 & yes \\
    57 & Homebrew &   6 & no \\
    58 & icsharpcode &   6 & no \\
    59 & msgpack &   6 & no \\
    60 & nodejs &   6 & no \\
    61 & openresty &   6 & no \\
    62 & ParsePlatform &   6 & yes \\
    63 & raspberrypi &   6 & partly \\
    64 & spotify &   6 & yes \\
    65 & twilio &   6 & yes \\
    66 & yahoo &   6 & yes \\
    67 & android &   5 & partly \\
    68 & angular &   5 & partly \\
    69 & angular-ui &   5 & partly \\
    70 & applidium &   5 & yes \\
    71 & celluloid &   5 & no \\
    72 & cesanta &   5 & yes \\
    73 & cucumber &   5 & yes \\
    74 & FriendsOfSymfony &   5 & no \\
    75 & heroku &   5 & yes \\
    76 & JetBrains &   5 & partly \\
    77 & paypal &   5 & yes \\
    78 & plataformatec &   5 & yes \\
    79 & RailsApps &   5 & no \\
    80 & rspec &   5 & no \\
    81 & ServiceStack &   5 & yes \\
    82 & socketio &   5 & no \\
    83 & ValveSoftware &   5 & yes \\
    84 & VerbalExpressions &   5 & no \\
    85 & visionmedia &   5 & no \\
    86 & activerecord-hackery &   4 & no \\
    87 & CakeDC &   4 & partly \\
    88 & castleproject &   4 & no \\
    89 & chef &   4 & yes \\
    90 & collectiveidea &   4 & yes \\
    91 & composer &   4 & no \\
    92 & docopt &   4 & no \\
    93 & documentcloud &   4 & partly \\
    94 & Flipboard &   4 & yes \\
    95 & forkingdog &   4 & no \\
    96 & getsentry &   4 & yes \\
    97 & gliderlabs &   4 & yes \\
    98 & gorilla &   4 & no \\
    99 & Instagram &   4 & yes \\
    100 & intridea &   4 & yes \\
    101 & mapbox &   4 & yes \\
    102 & mutualmobile &   4 & yes \\
    103 & openstack &   4 & yes \\
    104 & owncloud &   4 & yes \\
    105 & phacility &   4 & yes \\
    106 & Qihoo360 &   4 & yes \\
    107 & Reactive-Extensions &   4 & yes \\
    108 & sass &   4 & no \\
    109 & sourcegraph &   4 & yes \\
    110 & symfony &   4 & partly \\
    111 & tumblr &   4 & yes \\
    112 & venmo &   4 & yes \\
    113 & yhat &   4 & yes \\
   \hline
   \caption{All Organizations having at least 4 top repositories}
   \label{tbl:organizations_with_top_repos}

\end{longtable}


	\begin{landscape}
		\footnotesize{
			% latex table generated in R 3.2.2 by xtable 1.7-4 package
% Sun Oct 11 16:20:54 2015

\begin{longtable}{rlrrrrrrrrrrrrrr}
  % \centering
  % \footnotesize
  % \begin{tabular}
    \hline
   & Firm & Repos & JS & ObjC & Go & C & Python & Ruby & Java & PHP & C++ & C\# & CS & Scala & Hack \\
    \hline
  1 & google &  79 &   6 &  &  15 &   6 &   8 &   1 &  16 &   3 &  23 &   1 &  &  &  \\
    2 & facebook &  58 &  10 &   9 &  &   7 &   4 &  &  10 &   2 &  14 &  &  &  &   2 \\
    3 & twitter &  48 &   3 &   2 &  &   4 &  &   3 &   4 &  &   2 &  &  &  30 &  \\
    4 & Microsoft &  21 &   1 &  &  &   2 &  &  &  &  &   8 &  10 &  &  &  \\
    5 & github &  20 &   1 &   1 &   3 &   1 &  &   8 &  &  &  &   2 &   4 &  &  \\
    6 & alibaba &  20 &  &  &  &   4 &  &  &  12 &  &   4 &  &  &  &  \\
    7 & linkedin &  12 &   2 &   2 &  &  &  &  &   7 &  &  &  &   1 &  &  \\
    8 & aws \tablefootnote{by Amazon.com} &  11 &   1 &   1 &   2 &  &   1 &   2 &   1 &   2 &  &   1 &  &  &  \\
    9 & yahoo &   6 &   2 &  &   1 &   1 &  &  &   1 &  &  &  &  &   1 &  \\
    10 & groupon &   4 &  &  &  &  &  &  &  &  &  &  &   4 &  &  \\
    11 & NetEase &   2 &   1 &  &  &  &  &  &  &  &  &   1 &  &  &  \\
    12 & yandex &   2 &  &  &  &  &   1 &  &  &  &   1 &  &  &  &  \\
    13 & eBay &   2 &  &  &  &  &   2 &  &  &  &  &  &  &  &  \\
    14 & SonyWWS &   2 &  &  &  &  &  &  &  &  &  &   2 &  &  &  \\
    15 & sony &   1 &  &  &   1 &  &  &  &  &  &  &  &  &  &  \\
    16 & awslabs \tablefootnote{by Amazon.com} &   1 &  &  &  &   1 &  &  &  &  &  &  &  &  &  \\
    17 & intel-iot-devkit &   1 &  &  &  &   1 &  &  &  &  &  &  &  &  &  \\
    18 & forcedotcom \tablefootnote{by Salesforce} &   1 &  &  &  &  &  &  &   1 &  &  &  &  &  &  \\
    19 & amazonwebservices \tablefootnote{by Amazon.com} &   1 &  &  &  &  &  &  &  &   1 &  &  &  &  &  \\
    20 & developerforce \tablefootnote{by Salesforce} &   1 &  &  &  &  &  &  &  &  &  &   1 &  &  &  \\
     \hline
  % \end{tabular}
  \caption{Commercial most succesfull and most popular companies and their \textit{GitHub} projects' programming languages. \textit{Source: GitHub API, October 2015} }
  \label{tbl:commercial_successfull_firms_on_github_with_languages}
\end{longtable}

		}
	\end{landscape}

	{\footnotesize
		% latex table generated in R 3.2.2 by xtable 1.8-0 package
% Wed Feb 10 09:27:17 2016

\begin{longtable}{rlllrlr}


  \hline
 & Name & Name on GitHub & Public Repositories & Top Repositories & On GitHub since \\
  \hline
  \endhead
1 & Adafruit Industries & adafruit & 469 &   8 & 2010 \\
  2 & Airbnb & airbnb &  86 &  10 & 2011 \\
  3 & Alibaba & alibaba &  87 &  19 & 2012 \\
  4 & Apple & apple &  18 &   7 & 2015 \\
  5 & Applidium & applidium &  15 &   5 & 2010 \\
  6 & Automattic & Automattic & 279 &   7 & 2011 \\
  7 & Amazon Web Services & aws &  49 &  10 & 2012 \\
  8 & Microsoft Azure & Azure & 148 &   7 & 2014 \\
  9 & Bitly & bitly &  26 &   8 & 2010 \\
  10 & Cesanta Software & cesanta &  24 &   5 & 2013 \\
  11 & Chef Software, Inc. & chef & 284 &   4 & 2008 \\
  12 & Cloudera & cloudera & 133 &   6 & 2009 \\
  13 & Collective Idea & collectiveidea & 155 &   4 & 2008 \\
  14 & (Cucumber) & (cucumber) &  (52) &   (5) & 2010 \\
  15 & Docker & docker &  64 &  15 & 2013 \\
  16 & Douban Inc. & douban &  38 &   7 & 2011 \\
  17 & Dropbox & dropbox & 104 &   6 & 2011 \\
  18 & elastic & elastic & 107 &  15 & 2014 \\
  19 & (enormego) & (enormego) &  (26) &   (6) & 2009 \\
  20 & Etsy, Inc. & etsy &  56 &  10 & 2010 \\
  21 & Facebook & facebook & 149 &  56 & 2009 \\
  22 & Flipboard & Flipboard &  19 &   4 & 2010 \\
  23 & (Sentry) & (getsentry) &  (79) &   (4) & 2012 \\
  24 & GitHub & github & 127 &  16 & 2008 \\
  25 & (Glider Labs) & (gliderlabs) &  (19) &   (4) & 2014 \\
  26 & Google & google & 681 &  85 & 2012 \\
  27 & Google Samples & googlesamples & 184 &  10 & 2014 \\
  28 & HashiCorp & hashicorp &  79 &  15 & 2011 \\
  29 & Heroku & heroku & 488 &   5 & 2008 \\
  30 & id Software & id-Software &  18 &   7 & 2012 \\
  31 & Instagram & Instagram &  25 &   4 & 2011 \\
  32 & INTRIDEA Inc. & intridea & 107 &   4 & 2008 \\
  33 & KNP Labs & KnpLabs &  92 &   8 & 2010 \\
  34 & LinkedIn & linkedin &  87 &  10 & 2010 \\
  35 & Mapbox & mapbox & 558 &   4 & 2011 \\
  36 & Microsoft & Microsoft & 405 &  33 & 2013 \\
  37 & mongodb & mongodb &  54 &  16 & 2009 \\
  38 & Mutual Mobile & mutualmobile &  60 &   4 & 2009 \\
  39 & Netflix, Inc. & Netflix &  99 &  17 & 2011 \\
  40 & OpenStack & openstack & 665 &   4 & 2010 \\
  41 & ownCloud & owncloud &  95 &   4 & 2012 \\
  42 & Parse & ParsePlatform &  56 &   6 & 2011 \\
  43 & PayPal & paypal & 140 &   5 & 2010 \\
  44 & Phacility & phacility &   6 &   4 & 2013 \\
  45 & (Plataformatec) & (plataformatec) &  (23) &   (5) & 2009 \\
  46 & Qihoo 360 & Qihoo360 &  17 &   4 & 2013 \\
  47 & Cloud Programmability Group & Reactive-Extensions &  38 &   4 & 2011 \\
  48 & (ServiceStack) & (ServiceStack) &  (32) &   (5) & 2011 \\
  49 & Shopify & Shopify & 265 &   9 & 2008 \\
  50 & Sourcegraph & sourcegraph & 150 &   4 & 2013 \\
  51 & Spotify & spotify & 131 &   6 & 2010 \\
  52 & Square & square & 157 &  30 & 2009 \\
  53 & Stripe & stripe &  66 &   7 & 2011 \\
  54 & thoughtbot, inc. & thoughtbot & 221 &  23 & 2008 \\
  55 & Tumblr & tumblr &  34 &   4 & 2010 \\
  56 & Twilio & twilio &  47 &   6 & 2009 \\
  57 & Twitter, Inc. & twitter & 135 &  18 & 2009 \\
  58 & Valve Software & ValveSoftware &  17 &   5 & 2012 \\
  59 & Venmo & venmo &  73 &   4 & 2010 \\
  60 & Xamarin & xamarin &  78 &  10 & 2011 \\
  61 & Yahoo Inc. & yahoo & 330 &   6 & 2008 \\
  62 & Yalantis & Yalantis &  36 &  13 & 2011 \\
  63 & Yelp.com & Yelp & 149 &   8 & 2009 \\
  64 & yhat & yhat & 102 &   4 & 2012 \\
   \hline

   \caption{Finally selected commercial firms for observation. Bracketed firms are sorted out since they are not using an united webdomain. \\ \\
   \textbf{Data source:} \\
   \tiny
   \url{https://git.zeitpulse.com/philipp/masterthesis-data/raw/master/csv/organizations.csv} \\
   \url{https://git.zeitpulse.com/philipp/masterthesis-data/raw/master/csv/commercial_classification/commercial_classification_of_organizations.csv}
   }
   \label{tbl:selected_commercial_firms}

 \end{longtable}

	}

	\begin{landscape}
	\begin{table}[!h] \centering
		{\footnotesize
		% latex table generated in R 3.2.2 by xtable 1.8-0 package
% Wed Feb 10 19:54:22 2016


\begin{tabular}{rlrrlrrrrr}



  \hline
 & Firm & Rating & Ratings count & Level & Culture and Values & Career Opportunities & Work-Life-Balance & OS Repos. & Top Repos. count \\
  \hline
1 & elastic & 4.90 &  62 & Very Satisfied & 4.80 & 4.80 & 4.50 & 107 &  15 \\
  2 & Facebook & 4.50 & 1297 & Very Satisfied & 4.50 & 4.30 & 3.70 & 149 &  56 \\
  3 & Airbnb & 4.50 & 299 & Very Satisfied & 4.70 & 4.30 & 3.90 &  86 &  10 \\
  4 & Google & 4.40 & 4665 & Very Satisfied & 4.40 & 4.00 & 4.00 & 681 &  85 \\
  5 & Square & 4.40 & 155 & Very Satisfied & 4.30 & 4.00 & 4.10 & 157 &  30 \\
  6 & Google Samples & 4.40 & 4665 & Very Satisfied & 4.40 & 4.00 & 4.00 & 184 &  10 \\
  7 & LinkedIn & 4.40 & 1340 & Very Satisfied & 4.50 & 4.10 & 4.10 &  87 &  10 \\
  8 & Etsy, Inc. & 4.30 &  38 & Very Satisfied & 4.30 & 3.50 & 4.20 &  56 &  10 \\
  9 & Twitter, Inc. & 4.00 & 400 & Satisfied & 4.10 & 3.70 & 4.00 & 135 &  18 \\
  10 & mongodb & 4.00 & 135 & Satisfied & 4.00 & 4.00 & 3.80 &  54 &  16 \\
  11 & Microsoft & 3.90 & 12492 & Satisfied & 3.70 & 3.60 & 3.60 & 405 &  33 \\
  12 & Alibaba & 3.90 &  42 & Satisfied & 3.90 & 4.00 & 3.40 &  87 &  19 \\
  13 & Netflix, Inc. & 3.70 & 544 & Satisfied & 3.90 & 3.40 & 3.40 &  99 &  17 \\
  14 & Amazon Web Services & 3.40 & 7572 & OK & 3.30 & 3.40 & 2.70 &  49 &  10 \\
   \hline

\end{tabular}

		}
		\caption{Firm ratings by employees (\textit{Source: Glassdoor API, 25.01.2016})}
		\label{tbl:glassdoor_firm_ratings}
	\end{table}
	\end{landscape}

\clearpage
\subsection{Regression Tables for H 1.1}
\label{sec:regression_tables_h1.1}

	\begin{table}[!h] \centering
	  \scriptsize{
		  
% Table created by stargazer v.5.2 by Marek Hlavac, Harvard University. E-mail: hlavac at fas.harvard.edu
% Date and time: Tue, Mar 15, 2016 - 11:34:23
\begin{tabular}{@{\extracolsep{5pt}}lcccc}
\\[-1.8ex]\hline
\hline \\[-1.8ex]
 & \multicolumn{4}{c}{\textit{Dependent variable:}} \\
\cline{2-5}
\\[-1.8ex] & ext. commits & int. commits & ext. commits & int. commits \\
\\[-1.8ex] & (1.1.5) & (1.1.6) & (1.1.7) & (1.1.8)\\ 
\hline \\[-1.8ex]
 int. commits & 5.566$^{***}$ &  & 0.453$^{***}$ &  \\
  & (0.448) &  & (0.075) &  \\
  & & & & \\
 ext. commits &  & 0.009$^{***}$ &  & 0.156$^{***}$ \\
  &  & (0.001) &  & (0.026) \\
  & & & & \\
 Constant & $-$145.306 & 100.309$^{***}$ & 381.728$^{**}$ & 710.625$^{***}$ \\
  & (166.753) & (6.572) & (167.964) & (93.460) \\
  & & & & \\
\hline \\[-1.8ex]
Observations & 2,809 & 2,809 & 478 & 478 \\
R$^{2}$ & 0.052 & 0.052 & 0.071 & 0.071 \\
Adjusted R$^{2}$ & 0.052 & 0.052 & 0.069 & 0.069 \\
Residual Std. Error & 8,483.016 (df = 2807) & 347.859 (df = 2807) & 3,408.818 (df = 476) & 1,997.843 (df = 476) \\
F Statistic & 154.275$^{***}$ (df = 1; 2807) & 154.275$^{***}$ (df = 1; 2807) & 36.143$^{***}$ (df = 1; 476) & 36.143$^{***}$ (df = 1; 476) \\
\hline
\hline \\[-1.8ex]
\textit{Note:}  & \multicolumn{4}{r}{$^{*}$p$<$0.1; $^{**}$p$<$0.05; $^{***}$p$<$0.01} \\
\end{tabular}

	  }
		\caption{Impact of internal commits on external commits (and v.v.) on "Residual Projects" (Model 1.1.5 - 1.1.6) and "Top Projects" older than 1 year (Model 1.1.7 - 1.1.8)}
		\label{tbl:regression_table_1.1.5-1.1.7}
	\end{table}

	\begin{table}[!h] \centering
	  \scriptsize{
		  
% Table created by stargazer v.5.2 by Marek Hlavac, Harvard University. E-mail: hlavac at fas.harvard.edu
% Date and time: Tue, Mar 15, 2016 - 11:34:25
\begin{tabular}{@{\extracolsep{5pt}}lcccc}
\\[-1.8ex]\hline
\hline \\[-1.8ex]
 & \multicolumn{4}{c}{\textit{Dependent variable:}} \\
\cline{2-5}
\\[-1.8ex] & ext. commits & int. commits & ext. commits & int. commits \\
\\[-1.8ex] & (1.1.9) & (1.1.10) & (1.1.11) & (1.1.12)\\ 
\hline \\[-1.8ex]
 int. commits & 5.751$^{***}$ &  & 0.936$^{***}$ &  \\
  & (0.528) &  & (0.040) &  \\
  & & & & \\
 ext. commits &  & 0.010$^{***}$ &  & 0.864$^{***}$ \\
  &  & (0.001) &  & (0.037) \\
  & & & & \\
 Constant & $-$238.115 & 121.186$^{***}$ & $-$318.675 & 519.123$^{**}$ \\
  & (232.686) & (9.417) & (249.478) & (236.770) \\
  & & & & \\
\hline \\[-1.8ex]
Observations & 1,900 & 1,900 & 132 & 132 \\
R$^{2}$ & 0.059 & 0.059 & 0.809 & 0.809 \\
Adjusted R$^{2}$ & 0.058 & 0.058 & 0.807 & 0.807 \\
Residual Std. Error & 9,718.195 (df = 1898) & 409.982 (df = 1898) & 2,805.891 (df = 130) & 2,694.891 (df = 130) \\
F Statistic & 118.705$^{***}$ (df = 1; 1898) & 118.705$^{***}$ (df = 1; 1898) & 550.484$^{***}$ (df = 1; 130) & 550.484$^{***}$ (df = 1; 130) \\
\hline
\hline \\[-1.8ex]
\textit{Note:}  & \multicolumn{4}{r}{$^{*}$p$<$0.1; $^{**}$p$<$0.05; $^{***}$p$<$0.01} \\
\end{tabular}

	  }
		\caption{Impact of internal commits on external commits (and v.v.) on "Residual Projects" older than 1 year (Model 1.1.9 - 1.1.10) and "Top Projects" younger than 1 year (Model 1.1.11 - 1.1.12)}
		\label{tbl:regression_table_1.1.9-1.1.12}
	\end{table}

	\begin{table}[!h] \centering
	  \scriptsize{
		  
% Table created by stargazer v.5.2 by Marek Hlavac, Harvard University. E-mail: hlavac at fas.harvard.edu
% Date and time: Tue, Mar 15, 2016 - 11:34:26
\begin{tabular}{@{\extracolsep{5pt}}lcc}
\\[-1.8ex]\hline
\hline \\[-1.8ex]
 & \multicolumn{2}{c}{\textit{Dependent variable:}} \\
\cline{2-3}
\\[-1.8ex] & ext. commits & int. commits \\
\\[-1.8ex] & (1.1.13) & (1.1.14)\\ 
\hline \\[-1.8ex]
 int. commits & 2.323$^{*}$ &  \\
  & (1.225) &  \\
  & & \\
 ext. commits &  & 0.002$^{*}$ \\
  &  & (0.001) \\
  & & \\
 Constant & 190.823 & 58.289$^{***}$ \\
  & (180.326) & (4.483) \\
  & & \\
\hline \\[-1.8ex]
Observations & 909 & 909 \\
R$^{2}$ & 0.004 & 0.004 \\
Adjusted R$^{2}$ & 0.003 & 0.003 \\
Residual Std. Error (df = 907) & 4,983.821 & 134.877 \\
F Statistic (df = 1; 907) & 3.598$^{*}$ & 3.598$^{*}$ \\
\hline
\hline \\[-1.8ex]
\textit{Note:}  & \multicolumn{2}{r}{$^{*}$p$<$0.1; $^{**}$p$<$0.05; $^{***}$p$<$0.01} \\
\end{tabular}

	  }
		\caption{Impact of internal commits on external commits (and v.v.) on "Residual Projects" younger than 1 year (Model 1.1.13 - 1.1.14)}
		\label{tbl:regression_table_1.1.13-1.1.14}
	\end{table}


\clearpage
\subsection{Regression Tables for H 1.2.1}
\label{sec:regression_tables_h1.2.1}

	\begin{table}[!h] \centering
	  \scriptsize{
		  
% Table created by stargazer v.5.2 by Marek Hlavac, Harvard University. E-mail: hlavac at fas.harvard.edu
% Date and time: Tue, Mar 15, 2016 - 20:38:57
\begin{tabular}{@{\extracolsep{5pt}}lcccc}
\\[-1.8ex]\hline
\hline \\[-1.8ex]
 & \multicolumn{4}{c}{\textit{Dependent variable:}} \\
\cline{2-5}
\\[-1.8ex] & issues by ext. users & issues by firm empl. users & issues by ext. users & issues by firm empl. users \\
\\[-1.8ex] & (5) & (6) & (7) & (8)\\
\hline \\[-1.8ex]
 issues by firm empl. users & 5.211$^{***}$ &  & 34.605$^{***}$ &  \\
  & (0.448) &  & (2.815) &  \\
  & & & & \\
 issues by ext. users &  & 0.010$^{***}$ &  & 0.008$^{***}$ \\
  &  & (0.001) &  & (0.001) \\
  & & & & \\
 Constant & 40.005$^{***}$ & 0.357$^{***}$ & 242.496$^{***}$ & 2.892$^{***}$ \\
  & (2.717) & (0.125) & (47.815) & (0.724) \\
  & & & & \\
\hline \\[-1.8ex]
Observations & 2,412 & 2,412 & 414 & 414 \\
R$^{2}$ & 0.053 & 0.053 & 0.268 & 0.268 \\
Adjusted R$^{2}$ & 0.053 & 0.053 & 0.267 & 0.267 \\
Residual Std. Error & 132.237 (df = 2410) & 5.851 (df = 2410) & 898.292 (df = 412) & 13.449 (df = 412) \\
F Statistic & 135.324$^{***}$ (df = 1; 2410) & 135.324$^{***}$ (df = 1; 2410) & 151.164$^{***}$ (df = 1; 412) & 151.164$^{***}$ (df = 1; 412) \\
\hline
\hline \\[-1.8ex]
\textit{Note:}  & \multicolumn{4}{r}{$^{*}$p$<$0.1; $^{**}$p$<$0.05; $^{***}$p$<$0.01} \\
\end{tabular}

	  }
		\caption{Impact of issue participation by firm employed developers on external users (and v.v.) in "Residual Projects" (Model 1.2.5 - 1.2.6) and "Top Projects" older than 1 year (Model 1.2.7 - 1.2.8)}
	  \label{tbl:regression_table_1.2.5-1.2.8}
	\end{table}

	\begin{table}[!h] \centering
	  \scriptsize{
		  
% Table created by stargazer v.5.2 by Marek Hlavac, Harvard University. E-mail: hlavac at fas.harvard.edu
% Date and time: Tue, Mar 15, 2016 - 20:39:03
\begin{tabular}{@{\extracolsep{5pt}}lcccc}
\\[-1.8ex]\hline
\hline \\[-1.8ex]
 & \multicolumn{4}{c}{\textit{Dependent variable:}} \\
\cline{2-5}
\\[-1.8ex] & issues by ext. users & issues by firm empl. users & issues by ext. users & issues by firm empl. users \\
\\[-1.8ex] & (9) & (10) & (11) & (12)\\ 
\hline \\[-1.8ex]
 issues by firm empl. users & 5.051$^{***}$ &  & 15.993$^{***}$ &  \\
  & (0.526) &  & (3.744) &  \\
  & & & & \\
 issues by ext. users &  & 0.010$^{***}$ &  & 0.009$^{***}$ \\
  &  & (0.001) &  & (0.002) \\
  & & & & \\
 Constant & 47.408$^{***}$ & 0.543$^{***}$ & 179.710$^{***}$ & $-$0.124 \\
  & (3.813) & (0.180) & (33.079) & (0.892) \\
  & & & & \\
\hline \\[-1.8ex]
Observations & 1,679 & 1,679 & 109 & 109 \\
R$^{2}$ & 0.052 & 0.052 & 0.146 & 0.146 \\
Adjusted R$^{2}$ & 0.052 & 0.052 & 0.138 & 0.138 \\
Residual Std. Error & 154.470 (df = 1677) & 6.985 (df = 1677) & 338.349 (df = 107) & 8.075 (df = 107) \\
F Statistic & 92.310$^{***}$ (df = 1; 1677) & 92.310$^{***}$ (df = 1; 1677) & 18.248$^{***}$ (df = 1; 107) & 18.248$^{***}$ (df = 1; 107) \\
\hline
\hline \\[-1.8ex]
\textit{Note:}  & \multicolumn{4}{r}{$^{*}$p$<$0.1; $^{**}$p$<$0.05; $^{***}$p$<$0.01} \\
\end{tabular}

	  }
		\caption{Impact of issue participation by firm employed developers on external users (and v.v.) in "Residual Projects" older than 1 year (Model 1.2.9 - 1.2.10) and "Top Projects" younger than 1 year (Model 1.2.11 - 1.2.12)}
	  \label{tbl:regression_table_1.2.9-1.2.10}
	\end{table}

	\begin{table}[!h] \centering
	  % \scriptsize{
		  
% Table created by stargazer v.5.2 by Marek Hlavac, Harvard University. E-mail: hlavac at fas.harvard.edu
% Date and time: Tue, Mar 15, 2016 - 20:39:05
\begin{tabular}{@{\extracolsep{5pt}}lcc}
\\[-1.8ex]\hline
\hline \\[-1.8ex]
 & \multicolumn{2}{c}{\textit{Dependent variable:}} \\
\cline{2-3}
\\[-1.8ex] & issues by ext. users & issues by firm empl. users \\
\\[-1.8ex] & (13) & (14)\\ 
\hline \\[-1.8ex]
 issues by firm empl. users & 11.839$^{***}$ &  \\
  & (2.545) &  \\
  & & \\
 issues by ext. users &  & 0.002$^{***}$ \\
  &  & (0.001) \\
  & & \\
 Constant & 22.359$^{***}$ & 0.105$^{***}$ \\
  & (1.890) & (0.029) \\
  & & \\
\hline \\[-1.8ex]
Observations & 733 & 733 \\
R$^{2}$ & 0.029 & 0.029 \\
Adjusted R$^{2}$ & 0.027 & 0.027 \\
Residual Std. Error (df = 731) & 49.898 & 0.715 \\
F Statistic (df = 1; 731) & 21.636$^{***}$ & 21.636$^{***}$ \\
\hline
\hline \\[-1.8ex]
\textit{Note:}  & \multicolumn{2}{r}{$^{*}$p$<$0.1; $^{**}$p$<$0.05; $^{***}$p$<$0.01} \\
\end{tabular}

	  % }
		\caption{Impact of issue participation of firm employed users on external users (and v.v.) in "Residual Projects" younger than 1 year (Model 1.2.13 - 1.2.14)}
	  \label{tbl:regression_table_1.2.13-1.2.14}
	\end{table}


\clearpage
\subsection{Regression Tables for H 1.2.2}
\label{sec:regression_tables_h1.2.2}

\begin{landscape}
	\begin{table}[!h] \centering
	  \footnotesize{
		  
% Table created by stargazer v.5.2 by Marek Hlavac, Harvard University. E-mail: hlavac at fas.harvard.edu
% Date and time: Wed, Mar 16, 2016 - 01:26:39
\begin{tabular}{@{\extracolsep{5pt}}lcccc}
\\[-1.8ex]\hline
\hline \\[-1.8ex]
 & \multicolumn{4}{c}{\textit{Dependent variable:}} \\
\cline{2-5}
\\[-1.8ex] & Number of comments & \multicolumn{2}{c}{Comments by ext. developers} & Comments by firm employed developers \\
\\[-1.8ex] & (1.3.5) & (1.3.6) & (1.3.7) & (1.3.8)\\
\hline \\[-1.8ex]
 Content share by firm employed developers & $-$35,594.540 & $-$35,082.530 &  &  \\
  & (400,394.800) & (385,606.700) &  &  \\
  & & & & \\
 Comments by firm employed developers &  &  & 14.338$^{***}$ &  \\
  &  &  & (0.592) &  \\
  & & & & \\
 Comments by ext. developers &  &  &  & 0.038$^{***}$ \\
  &  &  &  & (0.002) \\
  & & & & \\
 Constant & 2,080.726$^{***}$ & 2,037.550$^{***}$ & 1,416.162$^{***}$ & $-$33.710$^{**}$ \\
  & (477.470) & (459.836) & (311.632) & (16.254) \\
  & & & & \\
\hline \\[-1.8ex]
Observations & 498 & 498 & 498 & 498 \\
R$^{2}$ & 0.00002 & 0.00002 & 0.542 & 0.542 \\
Adjusted R$^{2}$ & $-$0.002 & $-$0.002 & 0.541 & 0.541 \\
Residual Std. Error (df = 496) & 10,628.340 & 10,235.790 & 6,930.931 & 355.725 \\
F Statistic (df = 1; 496) & 0.008 & 0.008 & 585.804$^{***}$ & 585.804$^{***}$ \\
\hline
\hline \\[-1.8ex]
\textit{Note:}  & \multicolumn{4}{r}{$^{*}$p$<$0.1; $^{**}$p$<$0.05; $^{***}$p$<$0.01} \\
\end{tabular}

	  }
		\caption{Impact of participation by content share in issues' comments (Model 1.3.5 - 1.3.6) and numbers of comment by firm employed users on external users (and v.v.) (Model 1.3.7 - 1.3.8) in "Top Projects"}
		\label{tbl:regression_table_1.3.5-1.3.8}
	\end{table}


	\begin{table}[!h] \centering
	  \footnotesize{
		  
% Table created by stargazer v.5.2 by Marek Hlavac, Harvard University. E-mail: hlavac at fas.harvard.edu
% Date and time: Wed, Mar 16, 2016 - 01:27:25
\begin{tabular}{@{\extracolsep{5pt}}lcccc}
\\[-1.8ex]\hline
\hline \\[-1.8ex]
 & \multicolumn{4}{c}{\textit{Dependent variable:}} \\
\cline{2-5}
\\[-1.8ex] & Number of comments & \multicolumn{2}{c}{Comments by ext. developers} & Comments by firm employed developers \\
\\[-1.8ex] & (1.3.9) & (1.3.10) & (1.3.11) & (1.3.12)\\
\hline \\[-1.8ex]
 Content share by firm employed developers & 4,611.045 & 4,577.814 &  &  \\
  & (13,202.160) & (12,945.400) &  &  \\
  & & & & \\
 Comments by firm employed developers &  &  & 4.109$^{***}$ &  \\
  &  &  & (0.318) &  \\
  & & & & \\
 Comments by ext. developers &  &  &  & 0.018$^{***}$ \\
  &  &  &  & (0.001) \\
  & & & & \\
 Constant & 119.063$^{***}$ & 116.157$^{***}$ & 104.293$^{***}$ & 0.832 \\
  & (9.816) & (9.625) & (9.309) & (0.631) \\
  & & & & \\
\hline \\[-1.8ex]
Observations & 2,111 & 2,111 & 2,111 & 2,111 \\
R$^{2}$ & 0.0001 & 0.0001 & 0.073 & 0.073 \\
Adjusted R$^{2}$ & $-$0.0004 & $-$0.0004 & 0.073 & 0.073 \\
Residual Std. Error (df = 2109) & 450.873 & 442.104 & 425.589 & 28.056 \\
F Statistic (df = 1; 2109) & 0.122 & 0.125 & 166.993$^{***}$ & 166.993$^{***}$ \\
\hline
\hline \\[-1.8ex]
\textit{Note:}  & \multicolumn{4}{r}{$^{*}$p$<$0.1; $^{**}$p$<$0.05; $^{***}$p$<$0.01} \\
\end{tabular}

	  }
		\caption{Impact of participation by content share in issues' comments (Model 1.3.9 - 1.3.10) and numbers of comment by firm employed users on external users (and v.v.) (Model 1.3.11 - 1.3.12) in "Residual Projects"}
		\label{tbl:regression_table_1.3.9-1.3.12}
	\end{table}

\end{landscape}


\clearpage
\subsection{Regression Tables and Data Summary for H 2.1}
\label{sec:regression_tables_h2.1}

	\begin{table}[!h] \centering
		\footnotesize
		
% Table created by stargazer v.5.2 by Marek Hlavac, Harvard University. E-mail: hlavac at fas.harvard.edu
% Date and time: Wed, Mar 23, 2016 - 17:06:53
\begin{tabular}{@{\extracolsep{5pt}}lcccccc}
\\[-1.8ex]\hline
\hline \\[-1.8ex]
 & \multicolumn{6}{c}{\textit{Dependent variable:}} \\
\cline{2-7}
\\[-1.8ex] & \multicolumn{6}{c}{Top Project} \\
\\[-1.8ex] & (2.1.11) & (2.1.12) & (2.1.13) & (2.1.14) & (2.1.15) & (2.1.16)\\ 
\hline \\[-1.8ex]
 Age & 0.001$^{***}$ & 0.001$^{***}$ & 0.001$^{***}$ & 0.001$^{***}$ & 0.001$^{***}$ & 0.0002 \\
  & (0.0001) & (0.0001) & (0.0001) & (0.0001) & (0.0002) & (0.0002) \\
  & & & & & & \\
 $\text{Forks}_{1}$ & 2.125 &  &  &  &  &  \\
  & (1.419) &  &  &  &  &  \\
  & & & & & & \\
 $\text{Forks}_{2}$ &  & 3.020$^{***}$ &  &  &  &  \\
  &  & (0.711) &  &  &  &  \\
  & & & & & & \\
 $\text{Forks}_{3}$ &  &  & 2.883$^{***}$ &  &  &  \\
  &  &  & (0.532) &  &  &  \\
  & & & & & & \\
 $\text{Forks}_{4}$ &  &  &  & 0.884$^{**}$ &  &  \\
  &  &  &  & (0.397) &  &  \\
  & & & & & & \\
 $\text{Forks}_{5}$ &  &  &  &  & $-$2.017$^{***}$ &  \\
  &  &  &  &  & (0.315) &  \\
  & & & & & & \\
 forks.today &  &  &  &  &  & 0.007$^{***}$ \\
  &  &  &  &  &  & (0.001) \\
  & & & & & & \\
 Constant & $-$2.298$^{***}$ & $-$2.649$^{***}$ & $-$3.047$^{***}$ & $-$2.703$^{***}$ & $-$2.093$^{***}$ & $-$2.566$^{***}$ \\
  & (0.194) & (0.218) & (0.249) & (0.281) & (0.192) & (0.238) \\
  & & & & & & \\
\hline \\[-1.8ex]
Observations & 1,409 & 1,409 & 1,409 & 1,409 & 1,409 & 1,409 \\
Log Likelihood & $-$749.642 & $-$741.173 & $-$735.814 & $-$748.297 & $-$728.325 & $-$501.374 \\
Akaike Inf. Crit. & 1,505.285 & 1,488.346 & 1,477.627 & 1,502.593 & 1,462.650 & 1,008.749 \\
\hline
\hline \\[-1.8ex]
\textit{Note:}  & \multicolumn{6}{r}{$^{*}$p$<$0.1; $^{**}$p$<$0.05; $^{***}$p$<$0.01} \\
\end{tabular}

		\caption{The more forks a project gains in the beginning (i.e. first 3 - 6 months) the more likely it is a top project in the long-run (Model 2.1.11 - 2.1.16)}
		\label{tbl:regression_table_2.1.11-2.1.16}
	\end{table}

	\begin{table}[!h] \centering
		\footnotesize
		
% Table created by stargazer v.5.2 by Marek Hlavac, Harvard University. E-mail: hlavac at fas.harvard.edu
% Date and time: Wed, Mar 23, 2016 - 17:07:02
\begin{tabular}{@{\extracolsep{5pt}}lcccc}
\\[-1.8ex]\hline
\hline \\[-1.8ex]
 & \multicolumn{4}{c}{\textit{Dependent variable:}} \\
\cline{2-5}
\\[-1.8ex] & \multicolumn{4}{c}{Top Project} \\
\\[-1.8ex] & (2.1.17) & (2.1.18) & (2.1.19) & (2.1.20)\\ 
\hline \\[-1.8ex]
 Age & 0.001$^{***}$ & 0.001$^{***}$ & 0.001$^{***}$ & 0.001$^{***}$ \\
  & (0.0001) & (0.0001) & (0.0002) & (0.0001) \\
  & & & & \\
 $\text{Forks}_{1-2}$ & 5.695$^{***}$ &  &  &  \\
  & (1.271) &  &  &  \\
  & & & & \\
 $\text{Forks}_{2-3}$ &  & 5.199$^{***}$ &  &  \\
  &  & (0.804) &  &  \\
  & & & & \\
 $\text{Forks}_{3-4}$ &  &  & 3.445$^{***}$ &  \\
  &  &  & (0.684) &  \\
  & & & & \\
 $\text{Forks}_{4-5}$ &  &  &  & $-$5.106$^{***}$ \\
  &  &  &  & (0.773) \\
  & & & & \\
 Constant & $-$2.681$^{***}$ & $-$3.310$^{***}$ & $-$3.588$^{***}$ & $-$0.788$^{***}$ \\
  & (0.219) & (0.263) & (0.336) & (0.283) \\
  & & & & \\
\hline \\[-1.8ex]
Observations & 1,409 & 1,409 & 1,409 & 1,409 \\
Log Likelihood & $-$740.234 & $-$729.169 & $-$737.423 & $-$728.274 \\
Akaike Inf. Crit. & 1,486.467 & 1,464.338 & 1,480.846 & 1,462.547 \\
\hline
\hline \\[-1.8ex]
\textit{Note:}  & \multicolumn{4}{r}{$^{*}$p$<$0.1; $^{**}$p$<$0.05; $^{***}$p$<$0.01} \\
\end{tabular}

		\caption{If a projects gets forked in the beginning more often, the more likely it is a "Top Project" (Model 2.1.17 - 2.1.20)}
	  \label{tbl:regression_table_2.1.17-2.1.20}
	\end{table}

	\begin{table}[!h] \centering
		
% Table created by stargazer v.5.2 by Marek Hlavac, Harvard University. E-mail: hlavac at fas.harvard.edu
% Date and time: Wed, Mar 23, 2016 - 17:07:30
\begin{tabular}{@{\extracolsep{5pt}}lcccc}
\\[-1.8ex]\hline
\hline \\[-1.8ex]
 & \multicolumn{4}{c}{\textit{Dependent variable:}} \\
\cline{2-5}
\\[-1.8ex] & $\text{Forks}_{2}$ & $\text{Forks}_{3}$ & $\text{Forks}_{4}$ & $\text{Forks}_{5}$ \\
\\[-1.8ex] & (2.1.21) & (2.1.22) & (2.1.23) & (2.1.24)\\ 
\hline \\[-1.8ex]
 Age & $-$0.00003$^{***}$ & $-$0.0001$^{***}$ & $-$0.0002$^{***}$ & 0.0003$^{***}$ \\
  & (0.00001) & (0.00001) & (0.00001) & (0.00001) \\
  & & & & \\
 $\text{Ratio}_{1}$ & 0.031$^{***}$ &  &  &  \\
  & (0.007) &  &  &  \\
  & & & & \\
 $\text{Ratio}_{2}$ &  & 0.060$^{***}$ &  &  \\
  &  & (0.009) &  &  \\
  & & & & \\
 $\text{Ratio}_{3}$ &  &  & 0.058$^{***}$ &  \\
  &  &  & (0.012) &  \\
  & & & & \\
 $\text{Ratio}_{4}$ &  &  &  & $-$0.153$^{***}$ \\
  &  &  &  & (0.016) \\
  & & & & \\
 Constant & 0.113$^{***}$ & 0.238$^{***}$ & 0.486$^{***}$ & 0.168$^{***}$ \\
  & (0.007) & (0.010) & (0.014) & (0.019) \\
  & & & & \\
\hline \\[-1.8ex]
Observations & 1,409 & 1,409 & 1,409 & 1,409 \\
Log Likelihood & 1,407.889 & 1,003.989 & 574.908 & 127.614 \\
Akaike Inf. Crit. & $-$2,809.778 & $-$2,001.979 & $-$1,143.817 & $-$249.229 \\
\hline
\hline \\[-1.8ex]
\textit{Note:}  & \multicolumn{4}{r}{$^{*}$p$<$0.1; $^{**}$p$<$0.05; $^{***}$p$<$0.01} \\
\end{tabular}

		\caption{As the delay shows, number of "Forks" are rising relatively over time (see table \ref{tbl:time_observations_summary} for details). This maybe caused by network effects (the attention of a project in the beginning affects the later attention beneficial). Thus, further analyses would be necessary to proof the pure effect of "Ratio" on "Forks" over time. (Model 2.1.21 - 2.1.31).}
		\label{tbl:regression_table_2.1.21-2.1.31}
	\end{table}

	\begin{landscape}
	\begin{table}[!h] \centering
		\scriptsize{
			
% Table created by stargazer v.5.2 by Marek Hlavac, Harvard University. E-mail: hlavac at fas.harvard.edu
% Date and time: Wed, Mar 23, 2016 - 17:07:22
\begin{tabular}{@{\extracolsep{5pt}}lccccccccccc}
\\[-1.8ex]\hline
\hline \\[-1.8ex]
 & \multicolumn{11}{c}{\textit{Dependent variable:}} \\
\cline{2-12}
\\[-1.8ex] & $\text{Subscribers}_{1}$ & $\text{Subscribers}_{2}$ & $\text{Subscribers}_{3}$ & $\text{Subscribers}_{4}$ & $\text{Subscribers}_{2}$ & $\text{Subscribers}_{3}$ & $\text{Subscribers}_{4}$ & $\text{Subscribers}_{5}$ & $\text{Subscribers}_{3}$ & $\text{Subscribers}_{4}$ & $\text{Subscribers}_{5}$ \\
\\[-1.8ex] & (2.1.32) & (2.1.33) & (2.1.34) & (2.1.35) & (2.1.36) & (2.1.37) & (2.1.38) & (2.1.39) & (2.1.40) & (2.1.41) & (2.1.42)\\ 
\hline \\[-1.8ex]
 Age & $-$0.00000 & $-$0.00001$^{***}$ & $-$0.00005$^{***}$ & $-$0.0001$^{***}$ & $-$0.00001$^{***}$ & $-$0.00005$^{***}$ & $-$0.0001$^{***}$ & 0.0002$^{***}$ & $-$0.00005$^{***}$ & $-$0.0001$^{***}$ & 0.0002$^{***}$ \\
  & (0.00000) & (0.00000) & (0.00000) & (0.00001) & (0.00000) & (0.00000) & (0.00001) & (0.00001) & (0.00000) & (0.00001) & (0.00001) \\
  & & & & & & & & & & & \\
 $\text{Ratio}_{1}$ & 0.004$^{***}$ &  &  &  & 0.021$^{***}$ &  &  &  & 0.055$^{***}$ &  &  \\
  & (0.001) &  &  &  & (0.004) &  &  &  & (0.007) &  &  \\
  & & & & & & & & & & & \\
 $\text{Ratio}_{2}$ &  & 0.023$^{***}$ &  &  &  & 0.054$^{***}$ &  &  &  & 0.079$^{***}$ &  \\
  &  & (0.003) &  &  &  & (0.006) &  &  &  & (0.011) &  \\
  & & & & & & & & & & & \\
 $\text{Ratio}_{3}$ &  &  & 0.067$^{***}$ &  &  &  & 0.097$^{***}$ &  &  &  & $-$0.190$^{***}$ \\
  &  &  & (0.006) &  &  &  & (0.011) &  &  &  & (0.016) \\
  & & & & & & & & & & & \\
 $\text{Ratio}_{4}$ &  &  &  & 0.089$^{***}$ &  &  &  & $-$0.148$^{***}$ &  &  &  \\
  &  &  &  & (0.011) &  &  &  & (0.016) &  &  &  \\
  & & & & & & & & & & & \\
 Constant & 0.006$^{***}$ & 0.069$^{***}$ & 0.200$^{***}$ & 0.546$^{***}$ & 0.071$^{***}$ & 0.207$^{***}$ & 0.552$^{***}$ & 0.176$^{***}$ & 0.212$^{***}$ & 0.562$^{***}$ & 0.175$^{***}$ \\
  & (0.001) & (0.004) & (0.007) & (0.013) & (0.004) & (0.007) & (0.012) & (0.019) & (0.007) & (0.012) & (0.019) \\
  & & & & & & & & & & & \\
\hline \\[-1.8ex]
Observations & 1,409 & 1,409 & 1,409 & 1,409 & 1,409 & 1,409 & 1,409 & 1,409 & 1,409 & 1,409 & 1,409 \\
Log Likelihood & 4,296.766 & 2,453.283 & 1,565.419 & 701.254 & 2,446.640 & 1,539.586 & 705.850 & 108.943 & 1,535.701 & 690.719 & 133.812 \\
Akaike Inf. Crit. & $-$8,587.532 & $-$4,900.567 & $-$3,124.839 & $-$1,396.507 & $-$4,887.280 & $-$3,073.172 & $-$1,405.699 & $-$211.885 & $-$3,065.401 & $-$1,375.438 & $-$261.624 \\
\hline
\hline \\[-1.8ex]
\textit{Note:}  & \multicolumn{11}{r}{$^{*}$p$<$0.1; $^{**}$p$<$0.05; $^{***}$p$<$0.01} \\
\end{tabular}

		}
		\caption{Influence of "Ratio" on number of "Subscribers" (Model 2.1.32 - 2.1.42). See table \ref{tbl:regression_table_2.1.21-2.1.31} for possible explanation of effects ("Subscribers" behaves similar to "Forks")}
		\label{tbl:regression_table_2.1.32-2.1.42}

	\end{table}
	\end{landscape}

	\begin{table}[!h] \centering
		
% Table created by stargazer v.5.2 by Marek Hlavac, Harvard University. E-mail: hlavac at fas.harvard.edu
% Date and time: Wed, Mar 23, 2016 - 17:07:43
\begin{tabular}{@{\extracolsep{5pt}}lcccc}
\\[-1.8ex]\hline
\hline \\[-1.8ex]
 & \multicolumn{4}{c}{\textit{Dependent variable:}} \\
\cline{2-5}
\\[-1.8ex] & \multicolumn{4}{c}{$\text{Forks}_{today}$} \\
\\[-1.8ex] & (2.1.43) & (2.1.44) & (2.1.45) & (2.1.46)\\ 
\hline \\[-1.8ex]
 Age & 0.180$^{***}$ & 0.186$^{***}$ & 0.186$^{***}$ & 0.185$^{***}$ \\
  & (0.034) & (0.034) & (0.034) & (0.034) \\
  & & & & \\
 $\text{Ratio}_{1}$ & 254.674$^{***}$ &  &  &  \\
  & (46.734) &  &  &  \\
  & & & & \\
 $\text{Ratio}_{2}$ &  & 194.085$^{***}$ &  &  \\
  &  & (43.119) &  &  \\
  & & & & \\
 $\text{Ratio}_{3}$ &  &  & 217.558$^{***}$ &  \\
  &  &  & (41.383) &  \\
  & & & & \\
 $\text{Ratio}_{4}$ &  &  &  & 135.234$^{***}$ \\
  &  &  &  & (40.749) \\
  & & & & \\
 Constant & $-$95.681$^{**}$ & $-$102.676$^{**}$ & $-$121.572$^{**}$ & $-$107.743$^{**}$ \\
  & (46.130) & (46.954) & (47.354) & (48.734) \\
  & & & & \\
\hline \\[-1.8ex]
Observations & 1,409 & 1,409 & 1,409 & 1,409 \\
Log Likelihood & $-$10,916.200 & $-$10,920.850 & $-$10,917.220 & $-$10,925.430 \\
Akaike Inf. Crit. & 21,838.410 & 21,847.700 & 21,840.430 & 21,856.870 \\
\hline
\hline \\[-1.8ex]
\textit{Note:}  & \multicolumn{4}{r}{$^{*}$p$<$0.1; $^{**}$p$<$0.05; $^{***}$p$<$0.01} \\
\end{tabular}

		\caption{The higher the code contribution share of firm employed developers in the beginning, the higher is the final number of forks (except between month 6 - 12) (Model 2.1.43 - 2.1.46)}
		\label{tbl:regression_table_2.1.43-2.1.46}
	\end{table}

	\begin{table}[!h] \centering
		
% Table created by stargazer v.5.2 by Marek Hlavac, Harvard University. E-mail: hlavac at fas.harvard.edu
% Date and time: Wed, Mar 23, 2016 - 17:07:45
\begin{tabular}{@{\extracolsep{5pt}}lcccc}
\\[-1.8ex]\hline
\hline \\[-1.8ex]
 & \multicolumn{4}{c}{\textit{Dependent variable:}} \\
\cline{2-5}
\\[-1.8ex] & \multicolumn{4}{c}{$\text{Subscribers}_{today}$} \\
\\[-1.8ex] & (2.1.47) & (2.1.48) & (2.1.49) & (2.1.50)\\
\hline \\[-1.8ex]
 Age & 0.405$^{**}$ & 0.430$^{***}$ & 0.424$^{***}$ & 0.434$^{***}$ \\
  & (0.163) & (0.163) & (0.163) & (0.163) \\
  & & & & \\
 $\text{Ratio}_{1}$ & 1,053.216$^{***}$ &  &  &  \\
  & (226.797) &  &  &  \\
  & & & & \\
 $\text{Ratio}_{2}$ &  & 877.369$^{***}$ &  &  \\
  &  & (208.852) &  &  \\
  & & & & \\
 $\text{Ratio}_{3}$ &  &  & 753.878$^{***}$ &  \\
  &  &  & (201.218) &  \\
  & & & & \\
 $\text{Ratio}_{4}$ &  &  &  & 728.765$^{***}$ \\
  &  &  &  & (197.006) \\
  & & & & \\
 Constant & 77.437 & 28.895 & 17.185 & $-$42.648 \\
  & (223.868) & (227.429) & (230.250) & (235.611) \\
  & & & & \\
\hline \\[-1.8ex]
Observations & 1,409 & 1,409 & 1,409 & 1,409 \\
Log Likelihood & $-$13,141.840 & $-$13,143.780 & $-$13,145.570 & $-$13,145.740 \\
Akaike Inf. Crit. & 26,289.680 & 26,293.550 & 26,297.130 & 26,297.480 \\
\hline
\hline \\[-1.8ex]
\textit{Note:}  & \multicolumn{4}{r}{$^{*}$p$<$0.1; $^{**}$p$<$0.05; $^{***}$p$<$0.01} \\
\end{tabular}

		\caption{The higher the code contribution share of firm employed developers in the beginning, the higher is the final number of forks (except between month 3 - 6) (Model 2.1.47 - 2.1.50)}
		\label{tbl:regression_table_2.1.47-2.1.50}
	\end{table}

	\begin{table}[!h] \centering
		% \footnotesize
		
% Table created by stargazer v.5.2 by Marek Hlavac, Harvard University. E-mail: hlavac at fas.harvard.edu
% Date and time: Wed, Mar 23, 2016 - 17:07:39
\begin{tabular}{@{\extracolsep{5pt}}lcccc}
\\[-1.8ex]\hline
\hline \\[-1.8ex]
 & \multicolumn{4}{c}{\textit{Dependent variable:}} \\
\cline{2-5}
\\[-1.8ex] & \multicolumn{4}{c}{$\text{Forks}_{today}$} \\
\\[-1.8ex] & (2.1.51) & (2.1.52) & (2.1.53) & (2.1.54)\\ 
\hline \\[-1.8ex]
 Age & 0.185$^{***}$ & 0.188$^{***}$ & 0.185$^{***}$ & 0.180$^{***}$ \\
  & (0.034) & (0.034) & (0.034) & (0.034) \\
  & & & & \\
 $\text{Ratio}_{1\_2}$ & 287.570$^{***}$ &  &  &  \\
  & (50.896) &  &  &  \\
  & & & & \\
 $\text{Ratio}_{2\_3}$ &  & 281.414$^{***}$ &  &  \\
  &  & (49.170) &  &  \\
  & & & & \\
 $\text{Ratio}_{3\_4}$ &  &  & 135.234$^{***}$ &  \\
  &  &  & (40.749) &  \\
  & & & & \\
 $\text{Ratio}_{4\_5}$ &  &  &  & 135.903$^{***}$ \\
  &  &  &  & (48.874) \\
  & & & & \\
 Constant & $-$114.287$^{**}$ & $-$133.835$^{***}$ & $-$107.743$^{**}$ & $-$108.402$^{**}$ \\
  & (46.716) & (47.585) & (48.734) & (50.104) \\
  & & & & \\
\hline \\[-1.8ex]
Observations & 1,409 & 1,409 & 1,409 & 1,409 \\
Log Likelihood & $-$10,915.110 & $-$10,914.710 & $-$10,925.430 & $-$10,927.070 \\
Akaike Inf. Crit. & 21,836.220 & 21,835.410 & 21,856.870 & 21,860.130 \\
\hline
\hline \\[-1.8ex]
\textit{Note:}  & \multicolumn{4}{r}{$^{*}$p$<$0.1; $^{**}$p$<$0.05; $^{***}$p$<$0.01} \\
\end{tabular}

		\caption{Code contribution share of firm employed developers in the first year has the biggest positive impact on the final number of forks (Model 2.1.51 - 2.1.54)}
		\label{tbl:regression_table_2.1.51-2.1.54}
	\end{table}

	\begin{table}[!h] \centering
		
% Table created by stargazer v.5.2 by Marek Hlavac, Harvard University. E-mail: hlavac at fas.harvard.edu
% Date and time: Wed, Mar 23, 2016 - 17:07:41
\begin{tabular}{@{\extracolsep{5pt}}lcccc}
\\[-1.8ex]\hline
\hline \\[-1.8ex]
 & \multicolumn{4}{c}{\textit{Dependent variable:}} \\
\cline{2-5}
\\[-1.8ex] & \multicolumn{4}{c}{$\text{Subscribers}_{today}$} \\
\\[-1.8ex] & (2.1.55) & (2.1.56) & (2.1.57) & (2.1.58)\\ 
\hline \\[-1.8ex]
 Age & 0.424$^{***}$ & 0.436$^{***}$ & 0.434$^{***}$ & 0.402$^{**}$ \\
  & (0.163) & (0.163) & (0.163) & (0.163) \\
  & & & & \\
 $\text{Ratio}_{1\_2}$ & 1,241.737$^{***}$ &  &  &  \\
  & (246.863) &  &  &  \\
  & & & & \\
 $\text{Ratio}_{2\_3}$ &  & 1,109.494$^{***}$ &  &  \\
  &  & (238.874) &  &  \\
  & & & & \\
 $\text{Ratio}_{3\_4}$ &  &  & 728.765$^{***}$ &  \\
  &  &  & (197.006) &  \\
  & & & & \\
 $\text{Ratio}_{4\_5}$ &  &  &  & 816.083$^{***}$ \\
  &  &  &  & (236.158) \\
  & & & & \\
 Constant & $-$10.923 & $-$64.506 & $-$42.648 & $-$81.107 \\
  & (226.587) & (231.175) & (235.611) & (242.100) \\
  & & & & \\
\hline \\[-1.8ex]
Observations & 1,409 & 1,409 & 1,409 & 1,409 \\
Log Likelihood & $-$13,140.000 & $-$13,141.840 & $-$13,145.740 & $-$13,146.610 \\
Akaike Inf. Crit. & 26,286.000 & 26,289.670 & 26,297.480 & 26,299.210 \\
\hline
\hline \\[-1.8ex]
\textit{Note:}  & \multicolumn{4}{r}{$^{*}$p$<$0.1; $^{**}$p$<$0.05; $^{***}$p$<$0.01} \\
\end{tabular}

		\caption{Code contribution share of firm employed developers in the first year has the biggest positive impact on the final number of subscribers (Model 2.1.55 - 2.1.58)}
		\label{tbl:regression_table_2.1.55-2.1.58}
	\end{table}

	\begin{table}[!h] \centering
		\footnotesize
		
% Table created by stargazer v.5.2 by Marek Hlavac, Harvard University. E-mail: hlavac at fas.harvard.edu
% Date and time: Wed, Mar 23, 2016 - 17:10:54
\begin{tabular}{@{\extracolsep{5pt}}lcccc}
\\[-1.8ex]\hline
\hline \\[-1.8ex]
 & \multicolumn{4}{c}{\textit{Dependent variable:}} \\
\cline{2-5}
\\[-1.8ex] & \multicolumn{4}{c}{Top Project} \\
\\[-1.8ex] & (2.1.59) & (2.1.60) & (2.1.61) & (2.1.62)\\ 
\hline \\[-1.8ex]
 Age & 0.001$^{***}$ & 0.001$^{***}$ & 0.001$^{***}$ & 0.001$^{***}$ \\
  & (0.0001) & (0.0001) & (0.0001) & (0.0001) \\
  & & & & \\
 $\text{Ratio}_{1-2}$ & 2.330$^{***}$ &  &  &  \\
  & (0.212) &  &  &  \\
  & & & & \\
 $\text{Ratio}_{2-3}$ &  & 2.408$^{***}$ &  &  \\
  &  & (0.213) &  &  \\
  & & & & \\
 $\text{Ratio}_{3-4}$ &  &  & 1.758$^{***}$ &  \\
  &  &  & (0.178) &  \\
  & & & & \\
 $\text{Ratio}_{4-5}$ &  &  &  & 2.488$^{***}$ \\
  &  &  &  & (0.230) \\
  & & & & \\
 Constant & $-$2.984$^{***}$ & $-$3.201$^{***}$ & $-$3.160$^{***}$ & $-$3.579$^{***}$ \\
  & (0.219) & (0.228) & (0.227) & (0.249) \\
  & & & & \\
\hline \\[-1.8ex]
Observations & 1,409 & 1,409 & 1,409 & 1,409 \\
Log Likelihood & $-$687.044 & $-$682.113 & $-$699.158 & $-$685.443 \\
Akaike Inf. Crit. & 1,380.088 & 1,370.227 & 1,404.317 & 1,376.885 \\
\hline
\hline \\[-1.8ex]
\textit{Note:}  & \multicolumn{4}{r}{$^{*}$p$<$0.1; $^{**}$p$<$0.05; $^{***}$p$<$0.01} \\
\end{tabular}

		\caption{The higher the code contribution share of firm employed developers in the beginning the more likely it is a "Top Project" in the long-run (Model 2.1.59 - 2.1.62)}
		\label{tbl:regression_table_2.1.59-2.1.62}
	\end{table}

	\begin{table}[!h] \centering
		{\footnotesize
			% Table created by stargazer v.5.2 by Marek Hlavac, Harvard University. E-mail: hlavac at fas.harvard.edu
% Date and time: Wed, Mar 23, 2016 - 17:05:23
% \begin{table}[!htbp] \centering
%   \caption{}
%   \label{}
\begin{tabular}{@{\extracolsep{5pt}}lccccc}
\\[-1.8ex]\hline
\hline \\[-1.8ex]
Statistic & \multicolumn{1}{c}{N} & \multicolumn{1}{c}{Mean} & \multicolumn{1}{c}{St. Dev.} & \multicolumn{1}{c}{Min} & \multicolumn{1}{c}{Max} \\
\hline \\[-1.8ex]
$\text{Forks}_{today}$ & 1,409 & 178.048 & 571.655 & 1 & 11,670 \\
$\text{Subscribers}_{today}$ & 1,409 & 773.444 & 2,744.954 & 1 & 56,785 \\
$\text{Number of Code Contributions}_{total}$ & 1,409 & 1,140,249.000 & 0.000 & 1,140,249 & 1,140,249 \\
$\text{Number of Code Contributions}_{1}$ & 1,409 & 13.781 & 61.867 & 0 & 1,040 \\
$\text{Number of Code Contributions}_{2}$ & 1,409 & 39.065 & 156.261 & 0 & 2,320 \\
$\text{Number of Code Contributions}_{3}$ & 1,409 & 85.478 & 326.570 & 0 & 4,476 \\
$\text{Number of Code Contributions}_{4}$ & 1,409 & 171.919 & 575.452 & 0 & 7,054 \\
$\text{Number of Code Contributions}_{5}$ & 1,409 & 499.018 & 2,836.235 & 0 & 65,040 \\
$\text{Ratio}_{1}$ & 1,409 & 0.169 & 0.320 & 0.000 & 1.000 \\
$\text{Ratio}_{2}$ & 1,409 & 0.221 & 0.348 & 0.000 & 1.000 \\
$\text{Ratio}_{3}$ & 1,409 & 0.283 & 0.361 & 0.000 & 1.000 \\
$\text{Ratio}_{4}$ & 1,409 & 0.359 & 0.370 & 0.000 & 1.000 \\
$\text{Ratio}_{5}$ & 1,409 & 0.474 & 0.359 & 0.000 & 1.000 \\
$\text{Subscribers}_{1}$ & 1,409 & 0.007 & 0.012 & 0.000 & 0.200 \\
$\text{Subscribers}_{2}$ & 1,409 & 0.060 & 0.043 & 0.000 & 0.333 \\
$\text{Subscribers}_{3}$ & 1,409 & 0.161 & 0.086 & 0.000 & 0.485 \\
$\text{Subscribers}_{4}$ & 1,409 & 0.392 & 0.165 & 0.000 & 1.000 \\
$\text{Subscribers}_{5}$ & 1,409 & 0.381 & 0.248 & 0.000 & 1.000 \\
$\text{Forks}_{1}$ & 1,409 & 0.012 & 0.040 & 0.000 & 1.000 \\
$\text{Forks}_{2}$ & 1,409 & 0.077 & 0.091 & 0.000 & 1.000 \\
$\text{Forks}_{3}$ & 1,409 & 0.162 & 0.125 & 0.000 & 1.000 \\
$\text{Forks}_{4}$ & 1,409 & 0.305 & 0.177 & 0.000 & 1.000 \\
$\text{Forks}_{5}$ & 1,409 & 0.445 & 0.257 & 0.000 & 1.000 \\
$\text{Ratio}_{1-2}$ & 1,409 & 0.195 & 0.293 & 0.000 & 1.000 \\
$\text{Ratio}_{2-3}$ & 1,409 & 0.252 & 0.304 & 0.000 & 1.000 \\
$\text{Ratio}_{3-4}$ & 1,409 & 0.359 & 0.370 & 0.000 & 1.000 \\
$\text{Ratio}_{4-5}$ & 1,409 & 0.416 & 0.308 & 0.000 & 1.000 \\
$\text{Subscribers}_{1-2}$ & 1,409 & 0.033 & 0.024 & 0.000 & 0.200 \\
$\text{Subscribers}_{2-3}$ & 1,409 & 0.110 & 0.060 & 0.000 & 0.333 \\
$\text{Subscribers}_{3-4}$ & 1,409 & 0.277 & 0.111 & 0.000 & 0.500 \\
$\text{Subscribers}_{4-5}$ & 1,409 & 0.386 & 0.062 & 0.167 & 0.500 \\
$\text{Forks}_{1-2}$ & 1,409 & 0.044 & 0.050 & 0.000 & 0.500 \\
$\text{Forks}_{2-3}$ & 1,409 & 0.119 & 0.085 & 0.000 & 0.500 \\
$\text{Forks}_{3-4}$ & 1,409 & 0.233 & 0.113 & 0.000 & 0.500 \\
$\text{Forks}_{4-5}$ & 1,409 & 0.375 & 0.088 & 0.000 & 0.500 \\
\hline \\[-1.8ex]
\end{tabular}
% \end{table}

		}
		\caption{Trend of social metrics and contribution ratio over time. All observed projects are at least 4 years old. Normalized values (between 0 - 1) for "Forks", "Subscribers" and "Ratio" are increasing over time in most cases.}
		\label{tbl:time_observations_summary}
	\end{table}



\clearpage


	\begin{landscape}
		\subsection{Regression Tables for H 3}
		\label{sec:regression_tables_h3}
		\tiny
		\centering
		\begin{longtable}{@{\extracolsep{5pt}}lcccccccccccc}

			
% Table created by stargazer v.5.2 by Marek Hlavac, Harvard University. E-mail: hlavac at fas.harvard.edu
% Date and time: Sun, Mar 20, 2016 - 23:41:57
% \begin{tabular}{@{\extracolsep{5pt}}lcccccccccccc}
\\[-1.8ex]\hline
\hline \\[-1.8ex]
 & \multicolumn{12}{c}{\textit{Dependent variable:}} \\
\cline{2-13}
\\[-1.8ex] & \multicolumn{4}{c}{Stars} & \multicolumn{4}{c}{Subscribers} & \multicolumn{4}{c}{Forks} \\
\\[-1.8ex] & (3.1) & (3.2) & (3.3) & (3.4) & (3.5) & (3.6) & (3.7) & (3.8) & (3.9) & (3.10) & (3.11) & (3.12)\\
\hline \\[-1.8ex]
\endhead
 Ratio & 295.609$^{***}$ & 278.398$^{***}$ & 129.899 & 152.407$^{*}$ & 7.184 & 7.068 & 18.616$^{***}$ & 17.985$^{***}$ & 51.697$^{***}$ & 39.361$^{**}$ & 26.293 & 29.970 \\
  & (75.069) & (79.011) & (83.121) & (83.466) & (6.010) & (6.284) & (6.284) & (6.322) & (18.318) & (19.329) & (20.865) & (21.002) \\
  & & & & & & & & & & & & \\
 Age & 0.367$^{***}$ & 0.437$^{***}$ & 0.443$^{***}$ & 0.494$^{***}$ & 0.031$^{***}$ & 0.036$^{***}$ & 0.037$^{***}$ & 0.039$^{***}$ & 0.129$^{***}$ & 0.143$^{***}$ & 0.151$^{***}$ & 0.158$^{***}$ \\
  & (0.047) & (0.049) & (0.050) & (0.051) & (0.004) & (0.004) & (0.004) & (0.004) & (0.011) & (0.012) & (0.013) & (0.013) \\
  & & & & & & & & & & & & \\
 lang\_C\# &  & $-$214.631 &  & 169.676 &  & $-$4.257 &  & 10.249 &  & 19.293 &  & 59.294 \\
  &  & (178.693) &  & (196.570) &  & (14.212) &  & (14.889) &  & (43.714) &  & (49.463) \\
  & & & & & & & & & & & & \\
 lang\_C++ &  & 116.238 &  & 332.343$^{**}$ &  & 1.946 &  & 25.198$^{**}$ &  & 33.039 &  & 83.661$^{**}$ \\
  &  & (167.244) &  & (166.608) &  & (13.301) &  & (12.620) &  & (40.913) &  & (41.924) \\
  & & & & & & & & & & & & \\
 lang\_Go &  & 239.094 &  & 334.448$^{*}$ &  & $-$4.575 &  & 12.997 &  & 41.503 &  & 59.721 \\
  &  & (172.819) &  & (183.499) &  & (13.745) &  & (13.899) &  & (42.277) &  & (46.174) \\
  & & & & & & & & & & & & \\
 lang\_Java &  & 157.940 &  & 282.315$^{*}$ &  & 25.440$^{**}$ &  & 25.214$^{**}$ &  & 77.153$^{**}$ &  & 109.166$^{***}$ \\
  &  & (155.887) &  & (159.385) &  & (12.398) &  & (12.073) &  & (38.135) &  & (40.106) \\
  & & & & & & & & & & & & \\
 lang\_JavaScript &  & 208.857 &  & 592.722$^{***}$ &  & 14.696 &  & 22.251$^{*}$ &  & 33.642 &  & 126.760$^{***}$ \\
  &  & (152.583) &  & (157.994) &  & (12.135) &  & (11.967) &  & (37.326) &  & (39.756) \\
  & & & & & & & & & & & & \\
 lang\_Objective-C &  & 362.103$^{*}$ &  & 363.975$^{*}$ &  & $-$0.915 &  & 7.949 &  & 44.242 &  & 72.779 \\
  &  & (187.742) &  & (191.544) &  & (14.932) &  & (14.509) &  & (45.927) &  & (48.198) \\
  & & & & & & & & & & & & \\
 lang\_PHP &  & $-$216.619 &  & 108.148 &  & 1.185 &  & 2.580 &  & $-$28.608 &  & 61.594 \\
  &  & (177.471) &  & (200.353) &  & (14.115) &  & (15.176) &  & (43.415) &  & (50.415) \\
  & & & & & & & & & & & & \\
 lang\_Python &  & $-$145.320 &  & 108.349 &  & $-$31.810$^{**}$ &  & $-$2.574 &  & $-$12.433 &  & 40.406 \\
  &  & (157.632) &  & (163.375) &  & (12.537) &  & (12.375) &  & (38.562) &  & (41.110) \\
  & & & & & & & & & & & & \\
 lang\_Ruby &  & $-$230.824 &  & $-$30.721 &  & $-$32.958$^{***}$ &  & $-$13.470 &  & $-$50.827 &  & 2.041 \\
  &  & (155.093) &  & (167.303) &  & (12.335) &  & (12.673) &  & (37.940) &  & (42.098) \\
  & & & & & & & & & & & & \\
 firm\_airbnb &  &  & 1,783.590$^{***}$ & 1,872.003$^{***}$ &  &  & 157.228$^{***}$ & 170.044$^{***}$ &  &  & 249.295$^{***}$ & 261.275$^{***}$ \\
  &  &  & (306.332) & (317.847) &  &  & (23.159) & (24.076) &  &  & (76.896) & (79.980) \\
  & & & & & & & & & & & & \\
 firm\_alibaba &  &  & 721.450$^{***}$ & 731.614$^{***}$ &  &  & 142.658$^{***}$ & 142.330$^{***}$ &  &  & 348.882$^{***}$ & 335.851$^{***}$ \\
  &  &  & (257.418) & (265.678) &  &  & (19.461) & (20.124) &  &  & (64.618) & (66.852) \\
  & & & & & & & & & & & & \\
 firm\_apple &  &  & 3,988.731$^{***}$ & 4,078.281$^{***}$ &  &  & 308.373$^{***}$ & 312.584$^{***}$ &  &  & 567.897$^{***}$ & 586.710$^{***}$ \\
  &  &  & (524.956) & (522.366) &  &  & (39.687) & (39.567) &  &  & (131.776) & (131.443) \\
  & & & & & & & & & & & & \\
 firm\_applidium &  &  & 444.278 & 427.531 &  &  & 20.273 & 28.301 &  &  & 37.993 & 41.116 \\
  &  &  & (655.790) & (660.620) &  &  & (49.578) & (50.039) &  &  & (164.618) & (166.231) \\
  & & & & & & & & & & & & \\
 firm\_Automattic &  &  & 347.271$^{*}$ & 285.364 &  &  & 153.892$^{***}$ & 159.146$^{***}$ &  &  & 33.362 & 13.352 \\
  &  &  & (183.921) & (207.865) &  &  & (13.904) & (15.745) &  &  & (46.168) & (52.305) \\
  & & & & & & & & & & & & \\
 firm\_aws &  &  & 450.729$^{*}$ & 492.689$^{*}$ &  &  & 47.916$^{**}$ & 57.181$^{***}$ &  &  & 122.577$^{*}$ & 125.413$^{*}$ \\
  &  &  & (259.128) & (270.254) &  &  & (19.590) & (20.471) &  &  & (65.047) & (68.004) \\
  & & & & & & & & & & & & \\
 firm\_Azure &  &  & 197.388 & 223.868 &  &  & 36.238$^{**}$ & 42.884$^{***}$ &  &  & 72.173 & 73.259 \\
  &  &  & (192.978) & (210.404) &  &  & (14.589) & (15.937) &  &  & (48.442) & (52.944) \\
  & & & & & & & & & & & & \\
 firm\_bitly &  &  & 91.677 & 114.545 &  &  & 17.437 & 27.925 &  &  & $-$60.302 & $-$40.775 \\
  &  &  & (380.401) & (388.841) &  &  & (28.758) & (29.453) &  &  & (95.489) & (97.844) \\
  & & & & & & & & & & & & \\
 firm\_cesanta &  &  & 378.819 & 532.016 &  &  & 26.698 & 40.404 &  &  & 95.627 & 144.460 \\
  &  &  & (391.038) & (402.779) &  &  & (29.562) & (30.509) &  &  & (98.160) & (101.351) \\
  & & & & & & & & & & & & \\
 firm\_chef &  &  & 57.765 & 342.192 &  &  & 41.216$^{***}$ & 71.395$^{***}$ &  &  & 10.695 & 75.783 \\
  &  &  & (180.801) & (212.299) &  &  & (13.669) & (16.081) &  &  & (45.385) & (53.421) \\
  & & & & & & & & & & & & \\
 firm\_cloudera &  &  & 158.901 & 171.780 &  &  & 23.838 & 21.199 &  &  & 36.376 & 11.930 \\
  &  &  & (249.194) & (260.622) &  &  & (18.839) & (19.741) &  &  & (62.553) & (65.580) \\
  & & & & & & & & & & & & \\
 firm\_collectiveidea &  &  & $-$197.327 & 39.931 &  &  & $-$45.061$^{**}$ & $-$16.310 &  &  & $-$113.033 & $-$54.816 \\
  &  &  & (283.407) & (303.640) &  &  & (21.426) & (23.000) &  &  & (71.142) & (76.405) \\
  & & & & & & & & & & & & \\
 firm\_docker &  &  & 1,833.834$^{***}$ & 1,803.627$^{***}$ &  &  & 154.281$^{***}$ & 161.640$^{***}$ &  &  & 418.454$^{***}$ & 429.360$^{***}$ \\
  &  &  & (272.022) & (293.341) &  &  & (20.565) & (22.219) &  &  & (68.284) & (73.813) \\
  & & & & & & & & & & & & \\
 firm\_douban &  &  & 208.868 & 305.613 &  &  & 15.831 & 31.079 &  &  & 23.678 & 45.856 \\
  &  &  & (306.898) & (314.065) &  &  & (23.201) & (23.789) &  &  & (77.039) & (79.028) \\
  & & & & & & & & & & & & \\
 firm\_dropbox &  &  & 469.635$^{*}$ & 600.255$^{**}$ &  &  & 35.674$^{*}$ & 50.410$^{**}$ &  &  & 62.852 & 84.987 \\
  &  &  & (260.330) & (269.009) &  &  & (19.681) & (20.376) &  &  & (65.349) & (67.690) \\
  & & & & & & & & & & & & \\
 firm\_elastic &  &  & 611.492$^{***}$ & 652.659$^{***}$ &  &  & 72.232$^{***}$ & 77.962$^{***}$ &  &  & 171.865$^{***}$ & 168.103$^{***}$ \\
  &  &  & (210.873) & (224.926) &  &  & (15.942) & (17.037) &  &  & (52.934) & (56.598) \\
  & & & & & & & & & & & & \\
 firm\_etsy &  &  & 752.616$^{***}$ & 809.325$^{***}$ &  &  & 42.131$^{**}$ & 52.976$^{**}$ &  &  & 75.912 & 80.411 \\
  &  &  & (277.259) & (287.605) &  &  & (20.961) & (21.785) &  &  & (69.598) & (72.370) \\
  & & & & & & & & & & & & \\
 firm\_facebook &  &  & 2,394.111$^{***}$ & 2,384.316$^{***}$ &  &  & 171.279$^{***}$ & 176.552$^{***}$ &  &  & 384.147$^{***}$ & 380.714$^{***}$ \\
  &  &  & (184.613) & (191.626) &  &  & (13.957) & (14.515) &  &  & (46.342) & (48.219) \\
  & & & & & & & & & & & & \\
 firm\_Flipboard &  &  & 3,845.847$^{***}$ & 3,760.156$^{***}$ &  &  & 182.437$^{***}$ & 183.684$^{***}$ &  &  & 388.654$^{**}$ & 363.904$^{**}$ \\
  &  &  & (656.115) & (659.318) &  &  & (49.602) & (49.941) &  &  & (164.700) & (165.904) \\
  & & & & & & & & & & & & \\
 firm\_github &  &  & 898.635$^{***}$ & 1,037.884$^{***}$ &  &  & 34.714$^{*}$ & 53.426$^{***}$ &  &  & 187.585$^{***}$ & 223.290$^{***}$ \\
  &  &  & (256.875) & (269.861) &  &  & (19.420) & (20.441) &  &  & (64.481) & (67.905) \\
  & & & & & & & & & & & & \\
 firm\_google &  &  & 720.257$^{***}$ & 727.488$^{***}$ &  &  & 53.652$^{***}$ & 58.067$^{***}$ &  &  & 133.897$^{***}$ & 130.516$^{***}$ \\
  &  &  & (146.826) & (155.824) &  &  & (11.100) & (11.803) &  &  & (36.857) & (39.210) \\
  & & & & & & & & & & & & \\
 firm\_googlesamples &  &  & 501.872$^{**}$ & 496.091$^{**}$ &  &  & 43.458$^{**}$ & 39.813$^{**}$ &  &  & 162.108$^{***}$ & 132.682$^{**}$ \\
  &  &  & (227.162) & (242.156) &  &  & (17.173) & (18.342) &  &  & (57.023) & (60.934) \\
  & & & & & & & & & & & & \\
 firm\_hashicorp &  &  & 814.807$^{***}$ & 791.444$^{***}$ &  &  & 44.781$^{**}$ & 53.007$^{**}$ &  &  & 109.553$^{*}$ & 123.786$^{*}$ \\
  &  &  & (254.954) & (280.208) &  &  & (19.275) & (21.225) &  &  & (63.999) & (70.508) \\
  & & & & & & & & & & & & \\
 firm\_heroku &  &  & 89.603 & 177.261 &  &  & 15.988 & 31.161$^{**}$ &  &  & 72.618$^{*}$ & 92.963$^{**}$ \\
  &  &  & (165.134) & (186.053) &  &  & (12.484) & (14.093) &  &  & (41.452) & (46.816) \\
  & & & & & & & & & & & & \\
 firm\_id-Software &  &  & 1,886.755 & 1,805.205 &  &  & 336.544$^{***}$ & 328.951$^{***}$ &  &  & 594.253 & 578.166 \\
  &  &  & (1,447.476) & (1,439.507) &  &  & (109.429) & (109.037) &  &  & (363.350) & (362.222) \\
  & & & & & & & & & & & & \\
 firm\_Instagram &  &  & 752.507 & 880.403 &  &  & 39.559 & 56.610 &  &  & 176.467 & 201.123 \\
  &  &  & (655.800) & (657.806) &  &  & (49.578) & (49.826) &  &  & (164.621) & (165.523) \\
  & & & & & & & & & & & & \\
 firm\_intridea &  &  & 181.327 & 321.635 &  &  & $-$24.038 & $-$1.633 &  &  & $-$36.053 & 2.153 \\
  &  &  & (308.743) & (323.098) &  &  & (23.341) & (24.473) &  &  & (77.502) & (81.301) \\
  & & & & & & & & & & & & \\
 firm\_KnpLabs &  &  & 9.014 & 137.359 &  &  & $-$0.787 & 12.789 &  &  & $-$16.603 & $-$12.197 \\
  &  &  & (223.086) & (267.556) &  &  & (16.865) & (20.266) &  &  & (56.000) & (67.325) \\
  & & & & & & & & & & & & \\
 firm\_linkedin &  &  & 325.648 & 280.647 &  &  & 24.240 & 26.410 &  &  & 33.529 & 14.243 \\
  &  &  & (241.938) & (252.207) &  &  & (18.291) & (19.104) &  &  & (60.732) & (63.463) \\
  & & & & & & & & & & & & \\
 firm\_mapbox &  &  & 185.569 & $-$48.125 &  &  & 80.295$^{***}$ & 78.695$^{***}$ &  &  & 26.587 & $-$14.349 \\
  &  &  & (151.975) & (170.937) &  &  & (11.489) & (12.948) &  &  & (38.149) & (43.013) \\
  & & & & & & & & & & & & \\
 firm\_Microsoft &  &  & 401.146$^{**}$ & 440.668$^{**}$ &  &  & 47.493$^{***}$ & 52.144$^{***}$ &  &  & 96.519$^{**}$ & 97.017$^{**}$ \\
  &  &  & (162.492) & (181.612) &  &  & (12.284) & (13.756) &  &  & (40.789) & (45.699) \\
  & & & & & & & & & & & & \\
 firm\_mongodb &  &  & 572.687$^{**}$ & 605.178$^{**}$ &  &  & 60.054$^{***}$ & 69.642$^{***}$ &  &  & 203.993$^{***}$ & 211.318$^{***}$ \\
  &  &  & (286.179) & (291.850) &  &  & (21.635) & (22.107) &  &  & (71.838) & (73.438) \\
  & & & & & & & & & & & & \\
 firm\_mutualmobile &  &  & 794.332$^{*}$ & 681.476 &  &  & 34.910 & 40.099 &  &  & 111.917 & 95.217 \\
  &  &  & (432.681) & (445.379) &  &  & (32.711) & (33.736) &  &  & (108.613) & (112.070) \\
  & & & & & & & & & & & & \\
 firm\_Netflix &  &  & 509.045$^{**}$ & 451.850$^{**}$ &  &  & 232.128$^{***}$ & 230.588$^{***}$ &  &  & 70.328 & 40.937 \\
  &  &  & (200.325) & (215.458) &  &  & (15.145) & (16.320) &  &  & (50.286) & (54.216) \\
  & & & & & & & & & & & & \\
 firm\_openstack &  &  & 162.883 & 354.423$^{*}$ &  &  & 13.372 & 35.155$^{**}$ &  &  & 64.127 & 100.846$^{**}$ \\
  &  &  & (166.677) & (188.503) &  &  & (12.601) & (14.278) &  &  & (41.840) & (47.433) \\
  & & & & & & & & & & & & \\
 firm\_owncloud &  &  & 245.363 & 293.141 &  &  & 94.300$^{***}$ & 104.324$^{***}$ &  &  & 84.892 & 82.323 \\
  &  &  & (220.658) & (242.050) &  &  & (16.682) & (18.334) &  &  & (55.390) & (60.907) \\
  & & & & & & & & & & & & \\
 firm\_ParsePlatform &  &  & 431.926 & 353.046 &  &  & 56.053$^{**}$ & 60.203$^{***}$ &  &  & 118.252 & 103.283 \\
  &  &  & (289.122) & (298.425) &  &  & (21.858) & (22.605) &  &  & (72.576) & (75.093) \\
  & & & & & & & & & & & & \\
 firm\_paypal &  &  & 88.295 & 56.179 &  &  & 9.617 & 15.543 &  &  & 16.517 & 4.902 \\
  &  &  & (196.640) & (210.941) &  &  & (14.866) & (15.978) &  &  & (49.361) & (53.079) \\
  & & & & & & & & & & & & \\
 firm\_phacility &  &  & 1,354.084$^{**}$ & 1,318.708$^{**}$ &  &  & 155.553$^{***}$ & 159.946$^{***}$ &  &  & 174.756 & 158.418 \\
  &  &  & (658.868) & (662.543) &  &  & (49.810) & (50.185) &  &  & (165.391) & (166.715) \\
  & & & & & & & & & & & & \\
 firm\_Qihoo360 &  &  & 943.963$^{*}$ & 1,053.359$^{*}$ &  &  & 122.062$^{***}$ & 126.461$^{***}$ &  &  & 317.453$^{**}$ & 339.558$^{**}$ \\
  &  &  & (558.211) & (557.352) &  &  & (42.201) & (42.217) &  &  & (140.124) & (140.246) \\
  & & & & & & & & & & & & \\
 firm\_Reactive-Extensions &  &  & 608.570$^{*}$ & 411.081 &  &  & 26.069 & 24.826 &  &  & 48.994 & 13.200 \\
  &  &  & (362.725) & (369.390) &  &  & (27.422) & (27.980) &  &  & (91.052) & (92.949) \\
  & & & & & & & & & & & & \\
 firm\_Shopify &  &  & 295.762 & 467.720$^{**}$ &  &  & 105.086$^{***}$ & 127.295$^{***}$ &  &  & 14.280 & 58.401 \\
  &  &  & (203.489) & (224.093) &  &  & (15.384) & (16.974) &  &  & (51.081) & (56.388) \\
  & & & & & & & & & & & & \\
 firm\_sourcegraph &  &  & 180.876 & 150.278 &  &  & 0.193 & 7.189 &  &  & 25.492 & 36.502 \\
  &  &  & (247.783) & (273.833) &  &  & (18.732) & (20.742) &  &  & (62.199) & (68.904) \\
  & & & & & & & & & & & & \\
 firm\_spotify &  &  & 264.807 & 328.380 &  &  & 45.336$^{***}$ & 51.815$^{***}$ &  &  & 33.513 & 31.464 \\
  &  &  & (227.333) & (238.452) &  &  & (17.186) & (18.062) &  &  & (57.066) & (60.002) \\
  & & & & & & & & & & & & \\
 firm\_square &  &  & 1,093.104$^{***}$ & 1,109.454$^{***}$ &  &  & 70.649$^{***}$ & 76.974$^{***}$ &  &  & 166.484$^{***}$ & 163.711$^{***}$ \\
  &  &  & (193.279) & (210.141) &  &  & (14.612) & (15.917) &  &  & (48.517) & (52.878) \\
  & & & & & & & & & & & & \\
 firm\_stripe &  &  & 238.656 & 323.727 &  &  & 29.079 & 44.894$^{**}$ &  &  & 23.239 & 45.943 \\
  &  &  & (284.560) & (295.828) &  &  & (21.513) & (22.408) &  &  & (71.431) & (74.439) \\
  & & & & & & & & & & & & \\
 firm\_thoughtbot &  &  & 514.152$^{***}$ & 684.286$^{***}$ &  &  & 9.095 & 32.676$^{**}$ &  &  & 29.911 & 73.026 \\
  &  &  & (195.882) & (219.319) &  &  & (14.809) & (16.613) &  &  & (49.171) & (55.187) \\
  & & & & & & & & & & & & \\
 firm\_tumblr &  &  & 439.802 & 445.527 &  &  & 43.491 & 52.326$^{*}$ &  &  & 43.340 & 43.519 \\
  &  &  & (360.833) & (369.461) &  &  & (27.279) & (27.985) &  &  & (90.577) & (92.967) \\
  & & & & & & & & & & & & \\
 firm\_twilio &  &  & 121.696 & 53.474 &  &  & 23.703 & 29.964 &  &  & 21.738 & 6.643 \\
  &  &  & (351.376) & (358.265) &  &  & (26.564) & (27.137) &  &  & (88.204) & (90.150) \\
  & & & & & & & & & & & & \\
 firm\_twitter &  &  & 779.237$^{***}$ & 762.750$^{***}$ &  &  & 166.364$^{***}$ & 172.280$^{***}$ &  &  & 69.571 & 64.321 \\
  &  &  & (230.516) & (240.739) &  &  & (17.427) & (18.235) &  &  & (57.865) & (60.577) \\
  & & & & & & & & & & & & \\
 firm\_ValveSoftware &  &  & 961.506$^{*}$ & 1,006.940$^{*}$ &  &  & 135.571$^{***}$ & 135.791$^{***}$ &  &  & 205.071 & 217.620 \\
  &  &  & (558.226) & (555.752) &  &  & (42.202) & (42.096) &  &  & (140.128) & (139.844) \\
  & & & & & & & & & & & & \\
 firm\_venmo &  &  & 436.297 & 462.792 &  &  & 34.925 & 47.095$^{**}$ &  &  & 46.361 & 53.875 \\
  &  &  & (302.225) & (315.025) &  &  & (22.848) & (23.862) &  &  & (75.865) & (79.269) \\
  & & & & & & & & & & & & \\
 firm\_xamarin &  &  & 143.419 & 239.294 &  &  & 127.241$^{***}$ & 134.535$^{***}$ &  &  & 184.614$^{***}$ & 193.921$^{***}$ \\
  &  &  & (265.400) & (298.732) &  &  & (20.064) & (22.628) &  &  & (66.622) & (75.170) \\
  & & & & & & & & & & & & \\
 firm\_yahoo &  &  & 291.255$^{*}$ & 88.268 &  &  & 15.581 & 13.410 &  &  & 45.036 & 4.177 \\
  &  &  & (172.914) & (190.246) &  &  & (13.072) & (14.410) &  &  & (43.405) & (47.871) \\
  & & & & & & & & & & & & \\
 firm\_Yalantis &  &  & 1,222.287$^{***}$ & 1,197.076$^{***}$ &  &  & 77.274$^{***}$ & 79.524$^{***}$ &  &  & 266.267$^{***}$ & 248.290$^{***}$ \\
  &  &  & (352.406) & (365.651) &  &  & (26.642) & (27.697) &  &  & (88.462) & (92.009) \\
  & & & & & & & & & & & & \\
 firm\_Yelp &  &  & 183.684 & 338.035 &  &  & 12.692 & 32.249$^{*}$ &  &  & 34.810 & 65.535 \\
  &  &  & (225.708) & (239.584) &  &  & (17.063) & (18.148) &  &  & (56.658) & (60.286) \\
  & & & & & & & & & & & & \\
 firm\_yhat &  &  & 710.520 & 767.158 &  &  & 29.902 & 44.637 &  &  & 84.530 & 97.330 \\
  &  &  & (471.268) & (476.262) &  &  & (35.628) & (36.075) &  &  & (118.299) & (119.841) \\
  & & & & & & & & & & & & \\
 Constant & $-$0.236 & $-$70.590 & $-$434.208$^{***}$ & $-$764.217$^{***}$ & 48.363$^{***}$ & 47.700$^{***}$ & $-$20.544$^{*}$ & $-$39.926$^{***}$ & $-$23.486 & $-$43.139 & $-$113.655$^{***}$ & $-$192.851$^{***}$ \\
  & (60.212) & (154.968) & (139.704) & (195.611) & (4.821) & (12.325) & (10.562) & (14.817) & (14.692) & (37.910) & (35.069) & (49.222) \\
  & & & & & & & & & & & & \\
\hline \\[-1.8ex]
Observations & 3,419 & 3,419 & 3,419 & 3,419 & 3,419 & 3,419 & 3,419 & 3,419 & 3,419 & 3,419 & 3,419 & 3,419 \\
Log Likelihood & $-$29,905.750 & $-$29,877.320 & $-$29,692.650 & $-$29,668.010 & $-$21,273.000 & $-$21,221.860 & $-$20,863.760 & $-$20,845.730 & $-$25,083.110 & $-$25,063.340 & $-$24,966.870 & $-$24,950.480 \\
Akaike Inf. Crit. & 59,817.500 & 59,778.650 & 59,505.300 & 59,474.010 & 42,552.000 & 42,467.720 & 41,847.520 & 41,829.460 & 50,172.220 & 50,150.680 & 50,053.740 & 50,038.950 \\
\hline
\hline \\[-1.8ex]
\textit{Note:}  & \multicolumn{12}{r}{$^{*}$p$<$0.1; $^{**}$p$<$0.05; $^{***}$p$<$0.01} \\
% \end{tabular}

		  \caption{Detailed Regression Table: Impact of "Age" and "Ratio" on "Stars", "Subscribers" and "Forks" through \textit{Fitting Generalized Linear Models} with dummy variables for \textit{Firms} and \textit{Programming Languages}. Using (a) dummy variables for each \textit{Firm} in model 3.3, 3.7 and 3.11 (b) dummy variables for each \textit{Language} in model 3.2, 3.6, 3.10 (c) dummy variables for each \textit{Firm and Language} in model 3.4, 3.8 and 3.12}
		  \label{tbl:statistics_glm_project_popularity}
		\end{longtable}
	\end{landscape}

	\begin{landscape}
		\scriptsize
		\centering
		\begin{longtable}{@{\extracolsep{5pt}}lcccccccccccc}

			
% Table created by stargazer v.5.2 by Marek Hlavac, Harvard University. E-mail: hlavac at fas.harvard.edu
% Date and time: Mon, Mar 21, 2016 - 00:04:00
% \begin{tabular}{@{\extracolsep{5pt}}lcccccccccccc}
\\[-1.8ex]\hline
\hline \\[-1.8ex]
 & \multicolumn{12}{c}{\textit{Dependent variable:}} \\
\cline{2-13}
\\[-1.8ex] & \multicolumn{4}{c}{Number of Issues} & \multicolumn{4}{c}{Number of Contributors} & \multicolumn{4}{c}{Top Project} \\
\\[-1.8ex] & \multicolumn{4}{c}{\textit{normal}} & \multicolumn{4}{c}{\textit{normal}} & \multicolumn{4}{c}{\textit{logistic}} \\
\\[-1.8ex] & (3.13) & (3.14) & (3.15) & (3.16) & (3.17) & (3.18) & (3.19) & (3.20) & (3.21) & (3.22) & (3.23) & (3.24)\\
\hline \\[-1.8ex]
\endhead
 ratio & 2.147 & 2.006 & 7.861$^{*}$ & 7.612$^{*}$ & $-$2.417 & $-$2.647 & 1.391 & 1.402 & 2.025$^{***}$ & 1.727$^{***}$ & 1.906$^{***}$ & 1.753$^{***}$ \\
  & (3.717) & (3.930) & (4.235) & (4.275) & (1.911) & (2.005) & (2.051) & (2.067) & (0.155) & (0.166) & (0.191) & (0.197) \\
  & & & & & & & & & & & & \\
 age & 0.015$^{***}$ & 0.017$^{***}$ & 0.021$^{***}$ & 0.022$^{***}$ & 0.018$^{***}$ & 0.019$^{***}$ & 0.020$^{***}$ & 0.021$^{***}$ & 0.001$^{***}$ & 0.001$^{***}$ & 0.001$^{***}$ & 0.001$^{***}$ \\
  & (0.002) & (0.002) & (0.003) & (0.003) & (0.001) & (0.001) & (0.001) & (0.001) & (0.0001) & (0.0001) & (0.0001) & (0.0001) \\
  & & & & & & & & & & & & \\
 lang\_C\# &  & 10.769 &  & 15.255 &  & $-$2.302 &  & $-$3.798 &  & $-$0.439 &  & 0.301 \\
  &  & (8.887) &  & (10.067) &  & (4.535) &  & (4.869) &  & (0.272) &  & (0.368) \\
  & & & & & & & & & & & & \\
 lang\_C++ &  & 14.470$^{*}$ &  & 17.171$^{**}$ &  & 2.457 &  & 6.712 &  & $-$0.320 &  & 0.555$^{*}$ \\
  &  & (8.318) &  & (8.533) &  & (4.244) &  & (4.127) &  & (0.247) &  & (0.307) \\
  & & & & & & & & & & & & \\
 lang\_Go &  & 22.068$^{**}$ &  & 14.360 &  & 6.397 &  & 2.801 &  & 0.151 &  & 0.391 \\
  &  & (8.595) &  & (9.398) &  & (4.386) &  & (4.545) &  & (0.252) &  & (0.321) \\
  & & & & & & & & & & & & \\
 lang\_Java &  & 14.278$^{*}$ &  & 17.304$^{**}$ &  & $-$5.369 &  & $-$3.103 &  & $-$0.678$^{***}$ &  & $-$0.717$^{**}$ \\
  &  & (7.753) &  & (8.163) &  & (3.956) &  & (3.948) &  & (0.232) &  & (0.285) \\
  & & & & & & & & & & & & \\
 lang\_JavaScript &  & 12.686$^{*}$ &  & 17.803$^{**}$ &  & $-$2.354 &  & 3.703 &  & $-$1.619$^{***}$ &  & $-$1.202$^{***}$ \\
  &  & (7.588) &  & (8.092) &  & (3.872) &  & (3.913) &  & (0.254) &  & (0.302) \\
  & & & & & & & & & & & & \\
 lang\_Objective-C &  & 4.762 &  & 5.631 &  & $-$6.667 &  & $-$4.925 &  & $-$0.120 &  & $-$0.463 \\
  &  & (9.337) &  & (9.810) &  & (4.765) &  & (4.744) &  & (0.280) &  & (0.348) \\
  & & & & & & & & & & & & \\
 lang\_PHP &  & 16.926$^{*}$ &  & 8.654 &  & $-$3.341 &  & $-$0.999 &  & $-$0.847$^{***}$ &  & $-$0.408 \\
  &  & (8.826) &  & (10.261) &  & (4.504) &  & (4.962) &  & (0.289) &  & (0.391) \\
  & & & & & & & & & & & & \\
 lang\_Python &  & 3.092 &  & 7.609 &  & 12.824$^{***}$ &  & 2.975 &  & $-$1.414$^{***}$ &  & $-$1.162$^{***}$ \\
  &  & (7.840) &  & (8.367) &  & (4.000) &  & (4.047) &  & (0.259) &  & (0.318) \\
  & & & & & & & & & & & & \\
 lang\_Ruby &  & $-$0.052 &  & 5.067 &  & $-$1.393 &  & $-$2.457 &  & $-$1.655$^{***}$ &  & $-$1.517$^{***}$ \\
  &  & (7.713) &  & (8.568) &  & (3.936) &  & (4.144) &  & (0.248) &  & (0.322) \\
  & & & & & & & & & & & & \\
 firm\_airbnb &  &  & 17.277 & 19.858 &  &  & 17.225$^{**}$ & 23.634$^{***}$ &  &  & 3.062$^{***}$ & 4.697$^{***}$ \\
  &  &  & (15.609) & (16.279) &  &  & (7.558) & (7.873) &  &  & (0.569) & (0.611) \\
  & & & & & & & & & & & & \\
 firm\_alibaba &  &  & 22.277$^{*}$ & 22.852$^{*}$ &  &  & 2.369 & 7.964 &  &  & 3.094$^{***}$ & 3.969$^{***}$ \\
  &  &  & (13.116) & (13.607) &  &  & (6.351) & (6.580) &  &  & (0.514) & (0.554) \\
  & & & & & & & & & & & & \\
 firm\_apple &  &  & 23.867 & 26.692 &  &  & 131.388$^{***}$ & 132.286$^{***}$ &  &  & 6.461$^{***}$ & 6.896$^{***}$ \\
  &  &  & (26.748) & (26.753) &  &  & (12.952) & (12.938) &  &  & (1.150) & (1.174) \\
  & & & & & & & & & & & & \\
 firm\_applidium &  &  & 5.866 & 12.765 &  &  & $-$3.683 & 4.679 &  &  & 3.300$^{***}$ & 4.150$^{***}$ \\
  &  &  & (33.415) & (33.834) &  &  & (16.180) & (16.362) &  &  & (0.992) & (1.037) \\
  & & & & & & & & & & & & \\
 firm\_Automattic &  &  & 37.468$^{***}$ & 38.024$^{***}$ &  &  & 11.555$^{**}$ & 15.210$^{***}$ &  &  & 1.441$^{***}$ & 2.388$^{***}$ \\
  &  &  & (9.371) & (10.646) &  &  & (4.538) & (5.148) &  &  & (0.559) & (0.620) \\
  & & & & & & & & & & & & \\
 firm\_aws &  &  & 18.783 & 20.934 &  &  & 15.376$^{**}$ & 21.263$^{***}$ &  &  & 2.162$^{***}$ & 3.118$^{***}$ \\
  &  &  & (13.203) & (13.841) &  &  & (6.393) & (6.694) &  &  & (0.539) & (0.580) \\
  & & & & & & & & & & & & \\
 firm\_Azure &  &  & 26.153$^{***}$ & 26.499$^{**}$ &  &  & 18.567$^{***}$ & 24.545$^{***}$ &  &  & 1.169$^{**}$ & 1.779$^{***}$ \\
  &  &  & (9.833) & (10.776) &  &  & (4.761) & (5.211) &  &  & (0.553) & (0.597) \\
  & & & & & & & & & & & & \\
 firm\_bitly &  &  & $-$7.492 & $-$3.982 &  &  & $-$6.882 & $-$4.807 &  &  & 2.773$^{***}$ & 3.309$^{***}$ \\
  &  &  & (19.383) & (19.914) &  &  & (9.386) & (9.631) &  &  & (0.646) & (0.684) \\
  & & & & & & & & & & & & \\
 firm\_cesanta &  &  & 15.082 & 23.531 &  &  & 17.458$^{*}$ & 21.512$^{**}$ &  &  & 2.835$^{***}$ & 3.213$^{***}$ \\
  &  &  & (19.925) & (20.628) &  &  & (9.648) & (9.976) &  &  & (0.692) & (0.729) \\
  & & & & & & & & & & & & \\
 firm\_chef &  &  & 22.877$^{**}$ & 31.331$^{***}$ &  &  & 16.439$^{***}$ & 23.568$^{***}$ &  &  & $-$0.379 & 1.402$^{**}$ \\
  &  &  & (9.212) & (10.873) &  &  & (4.461) & (5.258) &  &  & (0.641) & (0.697) \\
  & & & & & & & & & & & & \\
 firm\_cloudera &  &  & 9.800 & 8.119 &  &  & 6.025 & 12.201$^{*}$ &  &  & 1.217$^{**}$ & 2.206$^{***}$ \\
  &  &  & (12.697) & (13.348) &  &  & (6.148) & (6.455) &  &  & (0.592) & (0.625) \\
  & & & & & & & & & & & & \\
 firm\_collectiveidea &  &  & $-$10.260 & $-$2.599 &  &  & $-$9.977 & $-$3.410 &  &  & $-$0.080 & 1.598$^{**}$ \\
  &  &  & (14.440) & (15.551) &  &  & (6.992) & (7.521) &  &  & (0.681) & (0.731) \\
  & & & & & & & & & & & & \\
 firm\_docker &  &  & 100.533$^{***}$ & 101.238$^{***}$ &  &  & 61.459$^{***}$ & 63.974$^{***}$ &  &  & 3.918$^{***}$ & 4.346$^{***}$ \\
  &  &  & (13.860) & (15.023) &  &  & (6.712) & (7.266) &  &  & (0.531) & (0.585) \\
  & & & & & & & & & & & & \\
 firm\_douban &  &  & 3.401 & 8.936 &  &  & $-$0.517 & 2.190 &  &  & 2.503$^{***}$ & 3.583$^{***}$ \\
  &  &  & (15.637) & (16.085) &  &  & (7.572) & (7.779) &  &  & (0.615) & (0.661) \\
  & & & & & & & & & & & & \\
 firm\_dropbox &  &  & 18.458 & 23.230$^{*}$ &  &  & 11.241$^{*}$ & 14.929$^{**}$ &  &  & 2.229$^{***}$ & 3.268$^{***}$ \\
  &  &  & (13.265) & (13.777) &  &  & (6.423) & (6.663) &  &  & (0.601) & (0.642) \\
  & & & & & & & & & & & & \\
 firm\_elastic &  &  & 56.926$^{***}$ & 57.240$^{***}$ &  &  & 26.492$^{***}$ & 32.473$^{***}$ &  &  & 2.499$^{***}$ & 3.557$^{***}$ \\
  &  &  & (10.745) & (11.520) &  &  & (5.203) & (5.571) &  &  & (0.492) & (0.540) \\
  & & & & & & & & & & & & \\
 firm\_etsy &  &  & 8.063 & 10.800 &  &  & 7.771 & 13.172$^{*}$ &  &  & 2.540$^{***}$ & 3.837$^{***}$ \\
  &  &  & (14.127) & (14.730) &  &  & (6.841) & (7.123) &  &  & (0.545) & (0.587) \\
  & & & & & & & & & & & & \\
 firm\_facebook &  &  & 55.298$^{***}$ & 57.411$^{***}$ &  &  & 35.655$^{***}$ & 39.469$^{***}$ &  &  & 3.909$^{***}$ & 4.961$^{***}$ \\
  &  &  & (9.407) & (9.814) &  &  & (4.555) & (4.746) &  &  & (0.438) & (0.482) \\
  & & & & & & & & & & & & \\
 firm\_Flipboard &  &  & 26.656 & 28.141 &  &  & 18.759 & 26.441 &  &  & 5.318$^{***}$ & 6.601$^{***}$ \\
  &  &  & (33.431) & (33.767) &  &  & (16.188) & (16.330) &  &  & (1.197) & (1.241) \\
  & & & & & & & & & & & & \\
 firm\_github &  &  & 13.649 & 18.961 &  &  & 49.316$^{***}$ & 55.222$^{***}$ &  &  & 2.935$^{***}$ & 4.204$^{***}$ \\
  &  &  & (13.089) & (13.821) &  &  & (6.338) & (6.684) &  &  & (0.512) & (0.577) \\
  & & & & & & & & & & & & \\
 firm\_google &  &  & 37.187$^{***}$ & 38.032$^{***}$ &  &  & 16.524$^{***}$ & 19.939$^{***}$ &  &  & 2.894$^{***}$ & 3.743$^{***}$ \\
  &  &  & (7.481) & (7.981) &  &  & (3.623) & (3.859) &  &  & (0.407) & (0.438) \\
  & & & & & & & & & & & & \\
 firm\_googlesamples &  &  & 12.768 & 10.421 &  &  & 11.100$^{**}$ & 18.256$^{***}$ &  &  & 2.163$^{***}$ & 3.243$^{***}$ \\
  &  &  & (11.575) & (12.402) &  &  & (5.605) & (5.998) &  &  & (0.521) & (0.564) \\
  & & & & & & & & & & & & \\
 firm\_hashicorp &  &  & 53.445$^{***}$ & 54.075$^{***}$ &  &  & 32.248$^{***}$ & 35.202$^{***}$ &  &  & 3.328$^{***}$ & 3.662$^{***}$ \\
  &  &  & (12.991) & (14.351) &  &  & (6.290) & (6.940) &  &  & (0.510) & (0.580) \\
  & & & & & & & & & & & & \\
 firm\_heroku &  &  & 7.694 & 10.957 &  &  & 5.035 & 10.236$^{**}$ &  &  & 0.098 & 1.432$^{**}$ \\
  &  &  & (8.414) & (9.529) &  &  & (4.074) & (4.608) &  &  & (0.595) & (0.633) \\
  & & & & & & & & & & & & \\
 firm\_id-Software &  &  & $-$7.908 & $-$11.441 &  &  & $-$8.110 & $-$10.127 &  &  & 15.225 & 15.063 \\
  &  &  & (73.753) & (73.724) &  &  & (35.713) & (35.654) &  &  & (324.744) & (324.744) \\
  & & & & & & & & & & & & \\
 firm\_Instagram &  &  & 27.182 & 32.765 &  &  & 15.720 & 22.539 &  &  & 5.015$^{***}$ & 6.458$^{***}$ \\
  &  &  & (33.415) & (33.690) &  &  & (16.180) & (16.293) &  &  & (1.219) & (1.252) \\
  & & & & & & & & & & & & \\
 firm\_intridea &  &  & 2.502 & 8.195 &  &  & 3.275 & 9.105 &  &  & 1.146$^{*}$ & 2.856$^{***}$ \\
  &  &  & (15.731) & (16.547) &  &  & (7.618) & (8.003) &  &  & (0.682) & (0.729) \\
  & & & & & & & & & & & & \\
 firm\_KnpLabs &  &  & 5.080 & 9.419 &  &  & 7.771 & 13.198$^{**}$ &  &  & 1.429$^{***}$ & 2.141$^{***}$ \\
  &  &  & (11.367) & (13.703) &  &  & (5.504) & (6.627) &  &  & (0.548) & (0.642) \\
  & & & & & & & & & & & & \\
 firm\_linkedin &  &  & 8.258 & 7.944 &  &  & 3.720 & 8.295 &  &  & 2.126$^{***}$ & 3.300$^{***}$ \\
  &  &  & (12.328) & (12.917) &  &  & (5.969) & (6.247) &  &  & (0.541) & (0.579) \\
  & & & & & & & & & & & & \\
 firm\_mapbox &  &  & 18.398$^{**}$ & 15.918$^{*}$ &  &  & 8.194$^{**}$ & 9.834$^{**}$ &  &  & $-$0.084 & 1.051 \\
  &  &  & (7.744) & (8.755) &  &  & (3.750) & (4.234) &  &  & (0.635) & (0.663) \\
  & & & & & & & & & & & & \\
 firm\_Microsoft &  &  & 23.334$^{***}$ & 22.830$^{**}$ &  &  & 17.604$^{***}$ & 23.280$^{***}$ &  &  & 2.498$^{***}$ & 2.927$^{***}$ \\
  &  &  & (8.280) & (9.301) &  &  & (4.009) & (4.498) &  &  & (0.435) & (0.483) \\
  & & & & & & & & & & & & \\
 firm\_mongodb &  &  & 3.715 & 6.473 &  &  & 47.271$^{***}$ & 50.914$^{***}$ &  &  & 2.327$^{***}$ & 3.374$^{***}$ \\
  &  &  & (14.582) & (14.947) &  &  & (7.061) & (7.229) &  &  & (0.554) & (0.592) \\
  & & & & & & & & & & & & \\
 firm\_mutualmobile &  &  & 15.365 & 19.738 &  &  & 6.313 & 14.501 &  &  & 2.640$^{***}$ & 3.668$^{***}$ \\
  &  &  & (22.046) & (22.810) &  &  & (10.675) & (11.031) &  &  & (0.744) & (0.806) \\
  & & & & & & & & & & & & \\
 firm\_Netflix &  &  & 19.113$^{*}$ & 17.219 &  &  & 11.074$^{**}$ & 16.565$^{***}$ &  &  & 2.130$^{***}$ & 3.333$^{***}$ \\
  &  &  & (10.207) & (11.035) &  &  & (4.943) & (5.337) &  &  & (0.471) & (0.520) \\
  & & & & & & & & & & & & \\
 firm\_openstack &  &  & 6.710 & 13.118 &  &  & 63.274$^{***}$ & 66.251$^{***}$ &  &  & 0.421 & 1.962$^{***}$ \\
  &  &  & (8.493) & (9.654) &  &  & (4.112) & (4.669) &  &  & (0.639) & (0.689) \\
  & & & & & & & & & & & & \\
 firm\_owncloud &  &  & 82.412$^{***}$ & 85.365$^{***}$ &  &  & 23.018$^{***}$ & 28.135$^{***}$ &  &  & 1.799$^{***}$ & 2.691$^{***}$ \\
  &  &  & (11.243) & (12.397) &  &  & (5.444) & (5.995) &  &  & (0.655) & (0.710) \\
  & & & & & & & & & & & & \\
 firm\_ParsePlatform &  &  & 19.117 & 20.497 &  &  & 11.729 & 17.265$^{**}$ &  &  & 2.467$^{***}$ & 3.561$^{***}$ \\
  &  &  & (14.732) & (15.284) &  &  & (7.133) & (7.391) &  &  & (0.606) & (0.646) \\
  & & & & & & & & & & & & \\
 firm\_paypal &  &  & 4.931 & 5.371 &  &  & 5.610 & 10.504$^{**}$ &  &  & 0.746 & 1.847$^{***}$ \\
  &  &  & (10.019) & (10.803) &  &  & (4.852) & (5.225) &  &  & (0.598) & (0.634) \\
  & & & & & & & & & & & & \\
 firm\_phacility &  &  & 54.564 & 55.098 &  &  & 49.630$^{***}$ & 52.187$^{***}$ &  &  & 3.822$^{***}$ & 4.409$^{***}$ \\
  &  &  & (33.571) & (33.932) &  &  & (16.256) & (16.410) &  &  & (1.188) & (1.225) \\
  & & & & & & & & & & & & \\
 firm\_Qihoo360 &  &  & 27.084 & 31.405 &  &  & 8.399 & 11.151 &  &  & 4.385$^{***}$ & 4.713$^{***}$ \\
  &  &  & (28.443) & (28.545) &  &  & (13.773) & (13.805) &  &  & (0.889) & (0.896) \\
  & & & & & & & & & & & & \\
 firm\_Reactive-Extensions &  &  & 12.295 & 9.249 &  &  & 16.013$^{*}$ & 18.364$^{**}$ &  &  & 3.016$^{***}$ & 3.979$^{***}$ \\
  &  &  & (18.482) & (18.918) &  &  & (8.949) & (9.149) &  &  & (0.712) & (0.791) \\
  & & & & & & & & & & & & \\
 firm\_Shopify &  &  & 8.128 & 14.190 &  &  & 8.601$^{*}$ & 14.477$^{***}$ &  &  & 1.474$^{***}$ & 2.713$^{***}$ \\
  &  &  & (10.368) & (11.477) &  &  & (5.021) & (5.550) &  &  & (0.536) & (0.582) \\
  & & & & & & & & & & & & \\
 firm\_sourcegraph &  &  & 9.240 & 9.569 &  &  & 9.172 & 11.913$^{*}$ &  &  & 1.420$^{**}$ & 1.657$^{**}$ \\
  &  &  & (12.625) & (14.024) &  &  & (6.113) & (6.782) &  &  & (0.652) & (0.706) \\
  & & & & & & & & & & & & \\
 firm\_spotify &  &  & 10.709 & 12.302 &  &  & 13.646$^{**}$ & 19.293$^{***}$ &  &  & 1.394$^{**}$ & 2.593$^{***}$ \\
  &  &  & (11.583) & (12.212) &  &  & (5.609) & (5.906) &  &  & (0.580) & (0.619) \\
  & & & & & & & & & & & & \\
 firm\_square &  &  & 9.775 & 11.049 &  &  & 10.174$^{**}$ & 16.483$^{***}$ &  &  & 2.670$^{***}$ & 3.844$^{***}$ \\
  &  &  & (9.848) & (10.762) &  &  & (4.769) & (5.205) &  &  & (0.442) & (0.491) \\
  & & & & & & & & & & & & \\
 firm\_stripe &  &  & 1.525 & 5.807 &  &  & 13.700$^{*}$ & 18.910$^{***}$ &  &  & 2.186$^{***}$ & 3.363$^{***}$ \\
  &  &  & (14.499) & (15.151) &  &  & (7.021) & (7.327) &  &  & (0.582) & (0.628) \\
  & & & & & & & & & & & & \\
 firm\_thoughtbot &  &  & 4.963 & 11.521 &  &  & 14.751$^{***}$ & 21.141$^{***}$ &  &  & 1.889$^{***}$ & 3.558$^{***}$ \\
  &  &  & (9.981) & (11.232) &  &  & (4.833) & (5.432) &  &  & (0.470) & (0.534) \\
  & & & & & & & & & & & & \\
 firm\_tumblr &  &  & 8.418 & 12.004 &  &  & 5.231 & 12.260 &  &  & 2.643$^{***}$ & 3.808$^{***}$ \\
  &  &  & (18.386) & (18.922) &  &  & (8.903) & (9.151) &  &  & (0.718) & (0.756) \\
  & & & & & & & & & & & & \\
 firm\_twilio &  &  & 11.977 & 13.025 &  &  & 7.517 & 11.536 &  &  & 2.071$^{***}$ & 3.278$^{***}$ \\
  &  &  & (17.904) & (18.349) &  &  & (8.669) & (8.874) &  &  & (0.713) & (0.784) \\
  & & & & & & & & & & & & \\
 firm\_twitter &  &  & 10.452 & 11.960 &  &  & 4.574 & 9.325 &  &  & 2.334$^{***}$ & 3.544$^{***}$ \\
  &  &  & (11.746) & (12.329) &  &  & (5.687) & (5.963) &  &  & (0.486) & (0.525) \\
  & & & & & & & & & & & & \\
 firm\_ValveSoftware &  &  & 147.585$^{***}$ & 149.228$^{***}$ &  &  & 39.073$^{***}$ & 39.290$^{***}$ &  &  & 5.031$^{***}$ & 5.060$^{***}$ \\
  &  &  & (28.443) & (28.463) &  &  & (13.773) & (13.765) &  &  & (0.957) & (0.972) \\
  & & & & & & & & & & & & \\
 firm\_venmo &  &  & 11.621 & 17.151 &  &  & 9.068 & 16.387$^{**}$ &  &  & 2.216$^{***}$ & 3.375$^{***}$ \\
  &  &  & (15.399) & (16.134) &  &  & (7.457) & (7.803) &  &  & (0.670) & (0.724) \\
  & & & & & & & & & & & & \\
 firm\_xamarin &  &  & 8.392 & 7.062 &  &  & 8.570 & 17.036$^{**}$ &  &  & 2.387$^{***}$ & 2.533$^{***}$ \\
  &  &  & (13.523) & (15.300) &  &  & (6.548) & (7.399) &  &  & (0.571) & (0.652) \\
  & & & & & & & & & & & & \\
 firm\_yahoo &  &  & 12.268 & 9.708 &  &  & 8.536$^{**}$ & 11.367$^{**}$ &  &  & 0.693 & 1.850$^{***}$ \\
  &  &  & (8.811) & (9.743) &  &  & (4.266) & (4.712) &  &  & (0.598) & (0.634) \\
  & & & & & & & & & & & & \\
 firm\_Yalantis &  &  & 12.839 & 15.420 &  &  & 11.149 & 20.419$^{**}$ &  &  & 4.407$^{***}$ & 5.533$^{***}$ \\
  &  &  & (17.956) & (18.727) &  &  & (8.695) & (9.057) &  &  & (0.623) & (0.670) \\
  & & & & & & & & & & & & \\
 firm\_Yelp &  &  & 14.280 & 20.465$^{*}$ &  &  & 8.881 & 11.987$^{**}$ &  &  & 1.416$^{**}$ & 2.707$^{***}$ \\
  &  &  & (11.501) & (12.270) &  &  & (5.569) & (5.934) &  &  & (0.556) & (0.613) \\
  & & & & & & & & & & & & \\
 firm\_yhat &  &  & 37.694 & 41.471$^{*}$ &  &  & 12.393 & 14.404 &  &  & 2.774$^{***}$ & 4.315$^{***}$ \\
  &  &  & (24.013) & (24.392) &  &  & (11.628) & (11.796) &  &  & (0.807) & (0.836) \\
  & & & & & & & & & & & & \\
 Constant & 8.443$^{***}$ & $-$2.780 & $-$19.593$^{***}$ & $-$34.006$^{***}$ & 5.114$^{***}$ & 4.488 & $-$14.598$^{***}$ & $-$20.129$^{***}$ & $-$3.426$^{***}$ & $-$2.546$^{***}$ & $-$5.820$^{***}$ & $-$6.276$^{***}$ \\
  & (2.981) & (7.707) & (7.118) & (10.018) & (1.533) & (3.933) & (3.447) & (4.845) & (0.140) & (0.251) & (0.437) & (0.529) \\
  & & & & & & & & & & & & \\
\hline \\[-1.8ex]
Observations & 3,419 & 3,419 & 3,419 & 3,419 & 3,419 & 3,419 & 3,419 & 3,419 & 3,419 & 3,419 & 3,419 & 3,419 \\
Log Likelihood & $-$19,630.080 & $-$19,616.630 & $-$19,514.800 & $-$19,507.690 & $-$17,355.340 & $-$17,316.430 & $-$17,035.330 & $-$17,023.880 & $-$1,458.752 & $-$1,373.897 & $-$1,208.251 & $-$1,150.787 \\
Akaike Inf. Crit. & 39,266.160 & 39,257.260 & 39,149.610 & 39,153.370 & 34,716.680 & 34,656.870 & 34,190.660 & 34,185.760 & 2,923.504 & 2,771.795 & 2,536.502 & 2,439.574 \\
\hline
\hline \\[-1.8ex]
\textit{Note:}  & \multicolumn{12}{r}{$^{*}$p$<$0.1; $^{**}$p$<$0.05; $^{***}$p$<$0.01} \\
% \end{tabular}


			\caption{
			Detailed Regression Table: Impact of "Age" and "Ratio" on "Issues", "Contributors", "Popularity" and "Top Project" through \textit{Fitting Generalized Linear Models}. Using (a) dummy variables for each \textit{Firm}  in model 3.15, 3.19 and 3.23 (b) dummy variables for each \textit{Language} in model 3.14, 3.18, 3.22 (c) dummy variables for each \textit{Firm and Language} in model 3.16, 3.20 and 3.24}

		  \label{tbl:statistics_glm_project_size}
		\end{longtable}

	\end{landscape}

	\begin{landscape}
		\begin{table}
			\centering
			\footnotesize{
			  
% Table created by stargazer v.5.2 by Marek Hlavac, Harvard University. E-mail: hlavac at fas.harvard.edu
% Date and time: Mon, Mar 14, 2016 - 14:37:28
\begin{tabular}{@{\extracolsep{5pt}}lcccccc}
\\[-1.8ex]\hline
\hline \\[-1.8ex]
 & \multicolumn{6}{c}{\textit{Dependent variable:}} \\
\cline{2-7}
\\[-1.8ex] & Stars & Subscribers & Forks & Number of Issues & Number of Contributors & Top Project \\
\\[-1.8ex] & \textit{linear} & \textit{linear} & \textit{linear} & \textit{linear} & \textit{linear} & \textit{generalized linear} \\
 & \textit{mixed-effects} & \textit{mixed-effects} & \textit{mixed-effects} & \textit{mixed-effects} & \textit{mixed-effects} & \textit{mixed-effects} \\
\\[-1.8ex] & (3.31) & (3.32) & (3.33) & (3.34) & (3.35) & (3.36)\\
\hline \\[-1.8ex]
 Ratio & 168.200$^{**}$ & 17.839$^{***}$ & 33.219 & 5.788 & 1.305 & 1.765$^{***}$ \\
  & (82.283) & (6.258) & (20.382) & (4.133) & (2.042) & (0.189) \\
  & & & & & & \\
 Age & 0.466$^{***}$ & 0.038$^{***}$ & 0.149$^{***}$ & 0.020$^{***}$ & 0.021$^{***}$ & 0.001$^{***}$ \\
  & (0.050) & (0.004) & (0.012) & (0.002) & (0.001) & (0.0001) \\
  & & & & & & \\
 Constant & 102.770 & 43.266$^{***}$ & $-$18.116 & 2.346 & 1.793 & $-$3.393$^{***}$ \\
  & (119.455) & (10.364) & (24.220) & (4.530) & (3.133) & (0.315) \\
  & & & & & & \\
\hline \\[-1.8ex]
Observations & 3,419 & 3,419 & 3,419 & 3,419 & 3,419 & 3,419 \\
Log Likelihood & $-$29,760.380 & $-$20,956.550 & $-$25,023.000 & $-$19,581.640 & $-$17,133.280 & $-$1,252.915 \\
Akaike Inf. Crit. & 59,532.760 & 41,925.100 & 50,058.000 & 39,175.290 & 34,278.570 & 2,515.830 \\
Bayesian Inf. Crit. & 59,569.590 & 41,961.920 & 50,094.820 & 39,212.110 & 34,315.390 & 2,546.516 \\
\hline
\hline \\[-1.8ex]
\textit{Note:}  & \multicolumn{6}{r}{$^{*}$p$<$0.1; $^{**}$p$<$0.05; $^{***}$p$<$0.01} \\
\end{tabular}

			}
			\caption{Impact of "Age" and "Ratio" on projects' social metrics ("Stars", "Subscribers", "Forks" and "Issues") and popularity (is "Top Project" yes / no) through \textit{Mixed Model} between \textit{Firms} and \textit{Programming Languages}. "Number of Contributors" is just a control attribute and not a describing model.}
			\label{tbl:h3.1_mixed_models_size}
		\end{table}
	\end{landscape}



	\begin{table}
		\centering
		\footnotesize{
		  
% Table created by stargazer v.5.2 by Marek Hlavac, Harvard University. E-mail: hlavac at fas.harvard.edu
% Date and time: Mon, Mar 21, 2016 - 23:59:46
\begin{tabular}{@{\extracolsep{5pt}}lcccccc}
\\[-1.8ex]\hline
\hline \\[-1.8ex]
 & \multicolumn{6}{c}{\textit{Dependent variable:}} \\
\cline{2-7}
\\[-1.8ex] & \multicolumn{6}{c}{Ratio} \\
\\[-1.8ex] & (3.37) & (3.38) & (3.39) & (3.40) & (3.41) & (3.42)\\
\hline \\[-1.8ex]
 Stars & 0.00000 &  &  &  &  &  \\
  & (0.00000) &  &  &  &  &  \\
  & & & & & & \\
 Subscribers &  & 0.0001$^{**}$ &  &  &  &  \\
  &  & (0.00005) &  &  &  &  \\
  & & & & & & \\
 Forks &  &  & 0.00001 &  &  &  \\
  &  &  & (0.00001) &  &  &  \\
  & & & & & & \\
 Number of Issues &  &  &  & 0.0001 &  &  \\
  &  &  &  & (0.0001) &  &  \\
  & & & & & & \\
 Number of Contributors &  &  &  &  & $-$0.00004 &  \\
  &  &  &  &  & (0.0001) &  \\
  & & & & & & \\
 Top Project &  &  &  &  &  & 0.137$^{***}$ \\
  &  &  &  &  &  & (0.014) \\
  & & & & & & \\
 Constant & 0.549$^{***}$ & 0.542$^{***}$ & 0.550$^{***}$ & 0.550$^{***}$ & 0.552$^{***}$ & 0.515$^{***}$ \\
  & (0.024) & (0.024) & (0.024) & (0.024) & (0.024) & (0.023) \\
  & & & & & & \\
\hline \\[-1.8ex]
Observations & 3,419 & 3,419 & 3,419 & 3,419 & 3,419 & 3,419 \\
Log Likelihood & $-$819.663 & $-$814.992 & $-$818.802 & $-$816.630 & $-$816.721 & $-$767.020 \\
Akaike Inf. Crit. & 1,647.326 & 1,637.985 & 1,645.603 & 1,641.260 & 1,641.442 & 1,542.039 \\
Bayesian Inf. Crit. & 1,671.874 & 1,662.533 & 1,670.152 & 1,665.809 & 1,665.990 & 1,566.588 \\
\hline
\hline \\[-1.8ex]
\textit{Note:}  & \multicolumn{6}{r}{$^{*}$p$<$0.1; $^{**}$p$<$0.05; $^{***}$p$<$0.01} \\
\end{tabular}

		}
		\caption{Impact of projects' social metrics on "Ratio" using \textit{Mixed-Effects Models} between firms (Model 3.37 - 3.42)}
		\label{tbl:lmer_ratio_as_dependent_1}
	\end{table}

	\begin{table}
		\centering
		\footnotesize{
		  
% Table created by stargazer v.5.2 by Marek Hlavac, Harvard University. E-mail: hlavac at fas.harvard.edu
% Date and time: Tue, Mar 22, 2016 - 00:00:05
\begin{tabular}{@{\extracolsep{5pt}}lcccccc}
\\[-1.8ex]\hline
\hline \\[-1.8ex]
 & \multicolumn{6}{c}{\textit{Dependent variable:}} \\
\cline{2-7}
\\[-1.8ex] & \multicolumn{6}{c}{Ratio} \\
\\[-1.8ex] & (3.43) & (3.44) & (3.45) & (3.46) & (3.47) & (3.48)\\
\hline \\[-1.8ex]
 Stars & 0.00001 &  &  &  &  &  \\
  & (0.00000) &  &  &  &  &  \\
  & & & & & & \\
 Subscribers &  & 0.0001$^{**}$ &  &  &  &  \\
  &  & (0.00005) &  &  &  &  \\
  & & & & & & \\
 Forks &  &  & 0.00001 &  &  &  \\
  &  &  & (0.00001) &  &  &  \\
  & & & & & & \\
 Number of Issues &  &  &  & 0.0001 &  &  \\
  &  &  &  & (0.0001) &  &  \\
  & & & & & & \\
 Number of Contributors &  &  &  &  & $-$0.00004 &  \\
  &  &  &  &  & (0.0001) &  \\
  & & & & & & \\
 Top Repo &  &  &  &  &  & 0.125$^{***}$ \\
  &  &  &  &  &  & (0.014) \\
  & & & & & & \\
 Constant & 0.555$^{***}$ & 0.550$^{***}$ & 0.557$^{***}$ & 0.556$^{***}$ & 0.559$^{***}$ & 0.521$^{***}$ \\
  & (0.029) & (0.029) & (0.029) & (0.029) & (0.029) & (0.027) \\
  & & & & & & \\
\hline \\[-1.8ex]
Observations & 3,419 & 3,419 & 3,419 & 3,419 & 3,419 & 3,419 \\
Log Likelihood & $-$795.593 & $-$791.577 & $-$794.918 & $-$792.955 & $-$792.904 & $-$751.717 \\
Akaike Inf. Crit. & 1,601.185 & 1,593.154 & 1,599.837 & 1,595.910 & 1,595.808 & 1,513.433 \\
Bayesian Inf. Crit. & 1,631.871 & 1,623.839 & 1,630.522 & 1,626.595 & 1,626.494 & 1,544.119 \\
\hline
\hline \\[-1.8ex]
\textit{Note:}  & \multicolumn{6}{r}{$^{*}$p$<$0.1; $^{**}$p$<$0.05; $^{***}$p$<$0.01} \\
\end{tabular}

		}
		\caption{Impact of projects' social metrics on "Ratio" using \textit{Mixed-Effects Models} between firms and programming languages (Model 3.43 - 3.48)}
		\label{tbl:lmer_ratio_as_dependent_2}
	\end{table}

\normalsize
